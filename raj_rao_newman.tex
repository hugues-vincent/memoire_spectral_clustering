\documentclass{article} % article, report, book, slides, beamer, lettre, memoir
 
% =============================
% Import
\usepackage[francais]{babel}
\usepackage[T1]{fontenc}
\usepackage{fontspec} 
\newcommand\fr{\selectlanguage{francais}}
\usepackage{titlesec}
\usepackage{graphicx}
\usepackage{float}
\usepackage{amstext}
\usepackage{amssymb}
\usepackage{subcaption}
\usepackage{geometry}
\usepackage[colorlinks]{hyperref}
\usepackage{amsmath}
\usepackage{bbm}
\usepackage{blkarray}
 \geometry{
 a4paper,
 total={170mm,257mm},
 left=20mm,
 top=20mm,
 }


\graphicspath{ {./static/} }

\newtheorem{theorem}{Theorem}

\title{Graph spectra and the detectability of community structure in networks}
\author{Hugues-Vincent Ropert}
\date{Mai 2018}

% \newcommand{\example}{\textit{example} }

\titleformat{\chapter}
  {\Large\bfseries} % format
  {}                % label
  {0pt}             % sep
  {\huge}           % before-code


\begin{document}
\maketitle

\textbf{Todo: Chercher dans ``universality for generalized Wigner matrices with bernouilli distribution '' \& ``isotropic Semicircle Law and Deformation of Wigner Matrices'' des explications aux théorèmes --- 1.2}\\

\textbf{Todo: Lire article sur modularity matrix de newman et faire le lien avec la méthode spectrale --- 1.3}\\

\textbf{Todo: Lire article TIOMOKO COUILLET afin implémenter l'amélioration et expliquer les différences --- 3}\\
\tableofcontents
% \newpage

\part{Vue d'ensemble de la théorie du clustering spectral de graphes}
\section{Motivations}
\subsection{Introduction}
La théorie des réseaux, à pour but d'analyser les graphes correspondants à des réseaux tels que celui de l'Internet, de la politique ou de la classification en biologie.
Chacun de ces graphes ont des propriétés spécifiques telles que:
\begin{itemize}
 	\item[-] Centrality ;
 	\item[-] ``small world effect'' ;
 	\item[-] Clustering ;
 	\item[-] Efficiency ;
 	\item[-] Degree distribution ; 
 	\item[-] Community Structure.\\
 \end{itemize}
 C'est cette dernière propriété qui va nous intéresser, à savoir la structure de communauté d'un graphe.
 L'idée de communauté correspond à l'intuition selon laquelle il y a des nœuds qui ont un lien étroit et forment des sous ensembles de nœuds de ce graphe.
 Par exemple dans le graphe des représentants politique Français, les hommes politiques d'un même partie appartiendrait à une même communauté.
 Cependant, dans cette exemple les choses peuvent être plus subtiles que ça, c'est là qu'intervient la détection de communauté.

\subsection{La détection de communauté}
Il y a différentes classes de méthodes pour la détection de communauté, en voici une liste non exhaustive:
\begin{itemize}
	\item[A -] Graph partitioning
	\item[B -] Hierarchical clustering
	\item[C -] Partitional clustering
	\item[D -] Spectral clustering \\
\end{itemize}

Les techniques qui nous intéressent sont celles du clustering spectrale.
L'idée centrale de ce type de techniques est qu'un graphe est représentable par une matrice à partir de laquelle on peut utiliser les théories relatives à l'algèbre linéaire.

Un graphe $G$ est la donnée d'un couple $(V, E)$ tel que $V$ est une ensemble de nœuds et $E$ et un ensemble d'arêtes (i.e un couple $(i,j)$ où $i,j \in V$).
A partir de cette définition on peut représenter la graphe par une matrice dont les éléments correspondent à certaines données de $G$.
Il existe toute un ensemble de matrices représentant le graphe:
\begin{itemize}
  	\item[-] A: Adjencency Matrix ;
  	\item[-] L: Laplacian Matrix ;
  	\item[-] M: Modularity Matrix ;
  	\item[-] H: Bethe Hessian Matrix ;
  	\item[-] D: "$\alpha$-normalized” Adjacency Matrix.\\
\end{itemize}
La matrice d'adjacence d'un graphe, notée $A$, est définie telle que $\forall i,j \in V, \; A_{ij} = \mathbbm{1}_{(i,j) \in E}$.
Une colonne $i$ représente le nœud $i$ dont chaque composante, $j$, est égale à $1$ si il existe une arête entre $i$ et $j$ et $0$ sinon.\\
La procédure pour associer une communauté aux nœuds de $G$ via une méthode spectrale est la suivante: 
\begin{itemize}
	\item[1-] Calcul des vecteurs propres, $v_i$, de la matrice de représentation du Graphe $G$ ;
	\item[2-] Sélection des $l$ vecteurs propres portant l'information de la structure de communauté ; 
	\item[3-] Construction de la matrice $W = [v_1, \cdots, v_l] \in \mathbb{R}^{n\times l}$ ; 
	\item[4-] Projection des vecteur lignes $r_j$ de $W$ sur l'espace de dimension $l$, où chaque $r_j$ correspond  au nœud $j$ ; 
	\item[5-] Catégorisation des vecteurs $r_j $ dans une communauté via des algorithmes de clustering: \textit{K-Means}, \textit{Expetation-Maximization}, \textit{Support vector machine }, etc.\\
\end{itemize}

\par{\underline{Étape 1:}}
La matrice $A$ est carré, par conséquent elle peut être interprétée comme la représentation d'un endomorphisme dans un espace X de dimension $n$ (nombre de nœuds dans $G$).
Soit la base canonique $(e_i)_{i=1:n}$, chaque $e_i$ correspond au nœud $i$.
Les vecteurs propres de $A$ sont donc des combinaisons linéaires des nœuds de $G$.
Les nœuds concernés ont donc une certaine dépendance.
\par{\underline{Étape 2:}}
Les entrées de la Matrice $A$ sont modélisées par des variables aléatoires vérifiant certaines hypothèses (e.g centrées, moments finis, indépendantes ...).
La théorie des matrices aléatoires nous dit qu’asymptotiquement la mesure spectrale de A converge vers une loi déterministe $\mu$ (loi qui dépend de la matrice étudiée).
Les valeurs propres de $A$ distribuées selon $\mu$ peuvent être interpréter comme des fluctuations de la simulation aléatoire.\\
Cependant, lorsque le graphe admet une structure de communauté, les entrées ne sont plus tout à fait indépendantes.
Dans ce cas de figure, la théorie prévoie que des valeurs propres sortent du support de la mesure spectrale $\mu$.
Ce sont ces valeurs propres qui contiennent l'information de la structure de communauté de $G$.\\
Une remarque importante à faire est la suivante, si la structure de communauté de $G$ n'est pas assez ``explicite'', les valeurs propres censées porter l'information de ces dernières ne sortiront pas du support de la mesure spectrale théorique, et donc, seront interprétées comme du simple bruit.
Dans cette situation, la méthode spectrale est incapable de déceler la structure de communauté de $G$.
\par{\underline{Étape 4:}}
Les vecteurs colonnes de W correspondent à des combinaisons linéaires de nœuds.
De par la construction A, les entrées $i$ de ces vecteurs colonnes sont la somme des arêtes entre ces nœuds en dépendance linaire et le nœud $i$.
Les vecteurs propres que l'on a filtré sont ceux qui portent l'information des communautés.
Donc chaque vecteur ligne représente un nœud du graphe et l'espace dans lequel il se meut constitue l'information de la structure de communauté. 
\input{chapter/graph_spectral_clustering_perspectives}

\part{Article: Graph spectra and the detectability of community structure in networks}
% \part{partie}
% \chapter{chapitre}
% \section{section}
% \subsection{sous-section}
% \subsubsection{sous-sous-section}
% \paragraph{paragraphe}
% \subparagraph{sous-paragraphe}
\section{Théorie}
\subsection{Contexte}
Le but de ce papier est d'expliquer en détail la méthode de détection de communauté via la théorie spectrale des graphes de l'article de RR. Nadakuditi et M. E. J. Newmann .

Nous allons nous placer dans le contexte d'un Stochastic Block Model (SBM) à deux communautés (i.e \textbf{q}=2).
Autrement dit, dans un graphe \textbf{G} non orienté à \textbf{n} noeuds dont la probabilité d'existence d'une arête entre deux noeuds est la suivante:
\begin{itemize}
    \item[Même communauté:] $\mathbf{p_{in}}$  
    \item[Différentes communautés:] $\mathbf{p_{out}}$\\
\end{itemize}

Nous noterons \textbf{A} la matrice d'adjacence du graphe \textbf{G}.
Elle est symétrique de par le fait que le graphe soit non orienté.
Nous supposerons d'ailleurs, que ses éléments sont rangés dans l'ordre de leur communauté.
Dans notre cas avec \textbf{q}=2, les $\mathbf{n}/2$ premières lignes correspondent aux noeuds de la communauté 1, et les $\mathbf{n}/2$ dernières la communauté 2.
Même chose pour les colonnes, par symétrie de \textbf{A}.
Chaque élément de la matrice \textbf{A} est simulé par une loi de Bernoulli avec: 
\begin{equation} 
 A_{ij} \sim \left\{
  \begin{array}{lr}
    B(p_{in}) & : (i,j < \frac{n}{2}) \lor (i,j \ge \frac{n}{2}) \\
    B(p_{out}) & : else \; where
  \end{array}
\right.\nonumber
\end{equation}
\begin{equation} 
A_{ij} = A_{ji}\nonumber
\end{equation}


\subsection{Analyse spectrale de la matrice d'adjacence \textbf{A}}
L'idée générale de l'analyse spectrale qui va suivre est de nous ramener à un régime spectral de grande matrices aléatoires connu. 
Dans notre cas, nous verrons que le régime associé à notre matrice d'adjacence (SBM $\textbf{q}=2$) est celui du théorème de \textit{Wigner} avec perturbation de rang 1.
La trame sera la suivante:
\begin{itemize}
 	\item[1-] réécriture de la matrice $A = \langle A \rangle + X$;
 	\item[2-] étude de la mesure spectrale de X;
 	\item[3-] étude de la mesure spectrale de B tel que $B = X + P1$;
 	\item[4-] étude de la mesure spectrale de $A = B + P2 = X + P1 + P2$.
 \end{itemize} 
où $P1$, $P2$ sont des perturbations de rang 1 et $\langle A \rangle$ correspond à la moyenne de A du SBM.\\

Soit 
\begin{equation} 
\langle A \rangle = \frac{1}{2}(c_{in} + c_{out})\mathbf{11}^T + \frac{1}{2}(c_{in} - c_{out})\mathbf{uu}^T
\end{equation}
Avec
\begin{align*}
c_{in} &= np_{int} \\
c_{out} &= np_{out}\\
\mathbf{1} &= (1, 1, \ldots)/\sqrt{n}\\
\mathbf{u} &= (1, 1, \ldots, -1, -1, \ldots)/\sqrt{n}
\end{align*}

À présent \textbf{A} peut être écrite sous la forme $A = \langle A \rangle + X$.
X est interprétable comme la déviation entre la matrice d'adjacence du graphe et sa moyenne.
X est par définition une matrice aléatoire symétrique à entrées iid et de moyenne 0, elle est donc une matrice de Wigner.
Essayons d'analyser sa mesure spectrale.
On a 

\begin{equation}
X = A - \langle A \rangle\nonumber
\end{equation}
\begin{equation}
	X_{ij} \sim \left\{
	\begin{array}{lr}
		B(p_{in}) - p_{in} & : (i,j < \frac{n}{2}) \lor (i,j \ge \frac{n}{2}) \\
		B(p_{out}) - p_{out} & : else \; where
	\end{array}
\right.\nonumber
\end{equation}

Comme X est une matrice de Wigner, nous aimerions modifier la forme des entrées $X_{ij}$ en $\sigma_{ij} Z_{ij}$, où $Z_{ij}$ est une variable aléatoire centrée réduite, afin de nous ramener à des théorèmes connus.

pour $(i,j < \frac{n}{2}) \lor (i,j \ge \frac{n}{2}) $
\begin{align*}
\mathbb{E}(X_{ij}) &= \mathbb{E}(B(p_{in}))- p_{in} = 0\\
\sigma_{in}^2 &= \mathbb{V}(X_{ij}) \\ 
			  &= \mathbb{V}(B(p_{in})) \\
			  &= p_{in} (1 - p_{in})
\end{align*}
même raisonnement avec $p_{out}$ 
\begin{align*}
\sigma_{out}^2 =  p_{out} (1 - p_{out})
\end{align*}
finalement on obtient 
\subsection{Interprétation}

\section{Simulation}
Le but est de générer une matrice d'adjacence A sous les mêmes hypothèses que \eqref{eq:1} en faisant varier les différents paramètres, à savoir: $n$, $p_{in}$, $p_{out}$ \\

\begin{figure}[h]
\centering
\includegraphics[scale=0.6]{static/graph_n50_pin08_pout01.png}
\caption{Graphe généré à partir des paramètres: $n=50$, $p_{in}=0.8$, $p_{out}=0.1$}
\end{figure}

L'élément discriminant du test spectral étant la variable $ \Delta p= p_{in} - p_{out}$, nous allons tester 3 valeurs représentatives des différents types de résultats:
\begin{itemize}
	\item[1-] $\Delta p \in [-1,\: 0] \implies$ le graphe ne comporte pas de structure de communauté ;
	\item[2-] $\Delta p \in [0,\: p_{lim}] \implies$ le graphe comporte une structure de communauté mais la méthode spectrale ne réussi pas à la trouver ;
	\item[2-] $\Delta p \in [p_{lim},\: 1] \implies$ le graphe comporte une structure de communauté.\\
\end{itemize}

Par la suite nous allons tester pour les valeurs $\Delta p= 0.5, 0.08, -0.5$ .
De plus nous ferons une première vague de test avec $n=100$ et une autre avec $n=1000$.
La terminologie utilisée dans les figures ci-dessous est:
\begin{itemize}
	\item[- \underline{$n,\: p_{in},\: p_{out},\: p_{lim},\: z1,\: z2$}:] sont identiques aux notations utilisées jusqu'à présent ;
	\item[- \underline{$z1\: theoric, \:z2\: theoric$}:] sont les plus grandes valeurs propres $z1$ et $z2$ calculées via les équations \eqref{z1} et \eqref{z2} ;    
	\item[- \underline{$p_{in}\: estimated, \:p_{out}\: estimated$}:] sont les probabilités du SBM calculées a posteriori grâce aux valeurs propres $z1$ et $z2$.\\
\end{itemize}
Les $p_{in}\: estimated, \:p_{out}\: estimated$ sont calculés via un calcul d'optimisation (la fonction ``fsolve'' en python).
\begin{figure}[H]
	\begin{subfigure}{.5\textwidth}
		\centering
		\includegraphics[scale=0.58]{static/spectral_n100_pin08_pout03.png}
		\caption{$n=100$, $\Delta p=0.5$}
	\end{subfigure}
	\begin{subfigure}{.5\textwidth}
		\centering
		\includegraphics[scale=0.58]{static/spectral_n100_pin05_pout042.png}
		\caption{$n=100$, $\Delta p=0.08$}
	\end{subfigure}
	\begin{subfigure}{.5\textwidth}
		\centering
		\includegraphics[scale=0.58]{static/spectral_n100_pin02_pout07.png}
		\caption{$n=100$, $\Delta p=-0.5$}
	\end{subfigure}
	\begin{subfigure}{.5\textwidth}
		\centering
		\includegraphics[scale=0.58]{static/spectral_n1000_pin08_pout03.png}
		\caption{$n=1000$, $\Delta p=0.5$}
	\end{subfigure}
	\begin{subfigure}{.5\textwidth}
		\centering
		\includegraphics[scale=0.58]{static/spectral_n1000_pin05_pout042.png}
		\caption{$n=1000$, $\Delta p=0.08$}
	\end{subfigure}
	\begin{subfigure}{.5\textwidth}
		\centering
		\includegraphics[scale=0.58]{static/spectral_n1000_pin02_pout07.png}
		\caption{$n=1000$, $\Delta p=-0.5$}
	\end{subfigure}
\end{figure}

La première observation que l'on peut faire est que les valeurs propres de nos matrices d'adjacence sont bien distribuées selon la loi du demi-cercle de Wigner.
De plus, en fonction de $\Delta p$ la mesure spectrale est perturbé (ou pas) par une ou deux valeurs propres qui sortent du support de la distribution initiale.\\

Dans les cas avec $\Delta p = 0.5$ on voit très clairement deux valeurs propres qui se détachent du support de la distribution de Wigner.
Les valeurs $z1$, $z2$ correspondent bien aux valeurs théoriques et dans ce cas on retrouve, avec un taux d'erreur de l'ordre de $10^{-2}$, les valeurs $p_{in}$, $p_{out}$ du modèle utilisé pour générer le graphe.\\
Dans les cas avec $\Delta p = -0.5$ on ne s'attend pas à ce que les valeurs théorique du modèles correspondent aux valeurs trouvées par la simulation dans la mesure où il n'y a en théorie aucune structure de communauté dans le graphe.
Cependant, on observe valeur propre en dehors du support de la loi de Wigner négative.
Par conséquent lorsque l'on obtient une valeur propre négative, le modèle spectral définit ci-dessus nous permet de conclure qu'il n'y a pas de structure de communauté dans le graphe.\\
Enfin, dans le cas avec $\Delta p = 0.08$, il y a en théorie une structure de communauté. 
Cependant on se retrouve dans le cas limite décrit dans \autoref{subsec:1.3}  

\section{Généralisation}
\subsection{SBM avec 2 communautés de tailles différentes}

\subsection{SBM avec n communautés}

On peut a présent généraliser à un nombre de communautés $q \geq 2$.
Nous allons supposer que les communautés sont de même taille, à savoir $n_q = \frac{n}{q}$.
\paragraph{}\label{rq:contrainte model}
Une première contrainte apparaît, de par l'utilisation des théorèmes \ref{th:1} et \ref{th:2}, sur les valeurs des probabilités de la matrice d'adjacence $A$.
En effet, d'après le \autoref{th:1}, pour que la matrice $X$ est une mesure spectrale qui tende vers la loi de Wigner il faut que la norme 1 des vecteurs lignes de son profile de variance soient égales ($\parallel x \parallel_1 = \sum_{j=1}^{n}|x_j|$).
Par conséquent, si on veut augmenter le nombre de communautés $q$ dans le modèle, on est forcé de garder deux probabilités $p_{in}$ et $p_{out}$ qui jouent le même rôle que celles introduites précédemment (cf. \ref{rq:probability}).\\

On sait que la matrice d’adjacence du graphe sous le SBM à $q$ communautés est $A = X + \langle A \rangle$.  
Pour poursuivre l'analyse on va suivre la trame suivante:
\begin{itemize}
	\item[1-] Trouver l'équation de $\langle A \rangle$ ;
	\item[2-] Trouvez l'équation de X et déterminer son profile de variance ;
	\item[3-] Trouvez les $q$ valeurs propres associées aux perturbations de rang 1 ;
	\item[4-] Trouver $p_{lim}$.\\
\end{itemize}

% ------------------------------------------------- 1- trouver <A>  -------------------------------------------------
D'une manière générale on trouve que:
\begin{align} 
\langle A \rangle :&= n_q(p_{in} + (q-1)p_{out}) \mathbf{11}^T + n_q\frac{\sqrt{2(q-1)}}{\sqrt{q}}(p_{in}-p_{out})\sum_{i=1}^{q-1}\mathbf{u}_i\mathbf{u'}_i^T\\
				   &= \frac{c_{in} + (q-1)c_{out}}{q} \mathbf{11}^T + \frac{\sqrt{2(q-1)}}{q\sqrt{q}}(c_{in}-c_{out})\sum_{i=1}^{q-1}\mathbf{u}_i\mathbf{u'}_i^T
\end{align}
avec
\begin{align*}
\begin{bmatrix}
\mathbf{1} & \mathbf{u}_1 & \cdots & \mathbf{u}_{q - 1}
\end{bmatrix} 
&=
\begin{bmatrix}
\frac{\mathbf{v}_1}{\parallel \mathbf{v}_1 \parallel} & \frac{\mathbf{v}_2}{\parallel \mathbf{v}_2 \parallel} & \cdots & \frac{\mathbf{v}_q}{\parallel \mathbf{v}_q \parallel}
\end{bmatrix}\\
\begin{bmatrix}
\mathbf{1} & \mathbf{u'}_1 & \cdots & \mathbf{u'}_{q - 1}
\end{bmatrix} 
&=
\begin{bmatrix}
\frac{\mathbf{v'}_1}{\parallel \mathbf{v'}_1 \parallel} & \frac{\mathbf{v'}_2}{\parallel \mathbf{v'}_2 \parallel} & \cdots & \frac{\mathbf{v'}_q}{\parallel \mathbf{v'}_q \parallel}
\end{bmatrix}
\end{align*}
où 
\begin{align*}
\begin{bmatrix}
\mathbf{v}_1 & \mathbf{v}_2 &  \cdots & \mathbf{v}_{q}
\end{bmatrix} 
&=
\begin{bmatrix}
1 & -1 & -1 & \cdots & -1 \\
1 & 1 & 0 & \cdots & 0 \\
\vdots & 0 & \ddots &  & \vdots \\
\vdots & \vdots &  & \ddots & 0 \\
1 & 0 & \cdots& \cdots & 1 \\
\end{bmatrix} \otimes \mathbf{1}_{n_q} \in \mathit{M}_q \otimes \mathit{M}_{n_q\times1}=\mathit{M}_{n\times q}\\
\begin{bmatrix}
\mathbf{v'}_1 & \mathbf{v'}_2 &  \cdots & \mathbf{v'}_{q}
\end{bmatrix} 
&= \frac{1}{n}
\begin{bmatrix}
1 & -1 & -1 & \cdots & -1 \\
1 & q -1 & -1 & \cdots & -1 \\
\vdots & -1 & \ddots &  & \vdots \\
\vdots & \vdots &  & \ddots & -1 \\
1 & -1 & \cdots& \cdots & q-1 \\
\end{bmatrix}\otimes \mathbf{1}_{n_q} \in \mathit{M}_q \otimes \mathit{M}_{n_q\times1}=\mathit{M}_{n\times q}\\
\end{align*}
\begin{align}\label{eq:norm generalize}
\parallel \mathbf{v}_1 \parallel^2 &= \frac{1}{n} & \; & \langle \mathbf{1}  \:|\: \mathbf{u}_i  \rangle = 0 \;,\forall i\nonumber\\
\parallel \mathbf{v}_i \parallel^2 &= \frac{2n}{q} & \; & \langle \mathbf{1} \:|\: \mathbf{u'}_i  \rangle = 0 \;,\forall i \\
\parallel \mathbf{v'}_i \parallel^2 &= \frac{q - 1}{n} & \; & \langle \mathbf{u}_i\:|\: \mathbf{u'}_i  \rangle = \sqrt{\frac{q}{2(q-1)}} \;,\forall i\nonumber\\
 \langle \mathbf{u}_i \:|\: \mathbf{u}_j  \rangle &= \frac{1}{2n} \;,\forall i \neq j & \; & \langle \mathbf{u'}_i \:|\: \mathbf{u'}_j  \rangle = \frac{1}{q - 1} \;,\forall i \neq j \nonumber\\
\langle \mathbf{u}_i \:|\: \mathbf{u'}_j  \rangle &= 0 \;,\forall i \neq j \nonumber
\end{align}\\

% ------------------------------------------------- 2- trouver X  -------------------------------------------------
De la même manière que dans \autoref{ch:Analyse spectrale de la matrice d'adjacence} on trouve:
\begin{equation}
	X_{ij} \sim \left\{
	\begin{array}{lr}
		\sigma_{in} Z_{ij} & : (i,j \in P_{in}) \\
		\sigma_{out} Z_{ij} & : (i,j \in P_{out})
	\end{array}
\right.\nonumber
\end{equation}
Où $Z_{ij} = \frac{B(p) - p}{\sqrt{p(1-p)}} \;\;avec \; p = p_{in} \lor p_{out}$\\
Soit le profile de variance de $\frac{X}{\sqrt{n}}$ et $\sigma^2$ la somme de n'importe quel de ces vecteurs lignes on obtient: 
\begin{align}
\label{eq:sigma2} 
\sigma^2 = \frac{\sigma_{in}^2 + (q-1)\sigma_{out}^2}{q}
\end{align}\\

% ------------------------------------------------- 3- trouver les vp -------------------------------------------------
On va à présent essayer de trouver les $q$ valeurs propres de $A$ associées au perturbations de rang 1. 
Repartons de $A = X + \langle A \rangle$.
\begin{align}
&\Leftrightarrow \frac{A}{\sqrt{n}}v =\lambda v \nonumber \\
&\Leftrightarrow (\Gamma - \lambda I)v =-\alpha \mathbf{11}^Tv - \beta \sum_{i=1}^{q-1}\mathbf{u}_i\mathbf{u'}_i^T \label{eq:generalize}
\end{align}
Pour trouver la valeur propre associée à $\mathbf{11}^T$ on multiplie par à gauche par $\mathbf{1}^T(\Gamma -\lambda I)^{-1}$ et on obtient:
\begin{align}
\eqref{eq:generalize} &\Leftrightarrow \mathbf{1}^Tv =-\alpha \mathbf{1}^T(\Gamma -\lambda I)^{-1}\mathbf{11}^Tv - \beta \mathbf{1}^T(\Gamma -\lambda I)^{-1}\sum_{i=1}^{q-1}\mathbf{u}_i\mathbf{u'}_i^Tv \nonumber\\
&\xrightarrow[n \to +\infty]{} 1 = -\alpha g_{wig}^{\sigma^2}(\lambda) \nonumber\\
&\Leftrightarrow 1 = \alpha \frac{\lambda + \sqrt{\lambda^2 - 4\sigma^2}}{2\sigma^2} \nonumber\\
&\Leftrightarrow \lambda = \frac{c_{in} + (q-1)c_{out}}{q\sqrt{n}} + \frac{q\sqrt{n}\sigma^2}{c_{in} + (q-1)c_{out}} \label{eq:z2 generalize}
\end{align}
Si on remplace $q$ par $2$ on retrouve l'équation \eqref{z2}.\\
Pour trouver les valeurs propres associées aux $\mathbf{u}_i\mathbf{u'}_i^T$ on multiplie par à gauche par $\mathbf{u'}_i^T(\Gamma -\lambda I)^{-1}$ et on obtient:
\begin{align}
\eqref{eq:generalize} &\Leftrightarrow \mathbf{u'}_i^Tv =-\alpha \mathbf{u'}_i(\Gamma -\lambda I)^{-1}\mathbf{11}^Tv - \beta \mathbf{u'}_i(\Gamma -\lambda I)^{-1}\sum_{i=1}^{q-1}\mathbf{u}_i\mathbf{u'}_i^Tv \nonumber\\
&\xrightarrow[n \to +\infty]{} \mathbf{u'}_i^Tv = -\beta(\langle \mathbf{u'}_i\:|\: \mathbf{u}_i \rangle\mathbf{u'}_i^Tv + \sum_{j \neq i}^{} \langle \mathbf{u'}_i\:|\: \mathbf{u}_j \rangle\mathbf{u'}_j^Tv)g_{wig}^{\sigma^2}(\lambda) \nonumber\\
\eqref{eq:norm generalize} &\Rightarrow 1 = -\beta  \sqrt{\frac{q}{2(q-1)}} g_{wig}^{\sigma^2}(\lambda) \nonumber\\
&\Leftrightarrow 1 = \beta \sqrt{\frac{q}{2(q-1)}} \frac{\lambda + \sqrt{\lambda^2 - 4\sigma^2}}{2\sigma^2} \nonumber\\
&\Leftrightarrow \lambda = \frac{c_{in} - c_{out}}{q\sqrt{n}} + \frac{q\sqrt{n}\sigma^2}{c_{in} - c_{out}}\label{eq:z1 generalize}
\end{align}
Si on remplace $q$ par $2$ on retrouve l'équation \eqref{z1}.\\
Nous noterons $z1 =$ \eqref{eq:z1 generalize}  et $z2 =$ \eqref{eq:z2 generalize} pour la suite.
Ces deux équations suffisent à retrouver les probabilités $p_{in}$ et $p_{out}$ du modèle à partir des valeurs propres empiriques (i.e calculées numériquement sur la matrice adjacente du graphe étudier.)\\

% ------------------------------------------------- 4- trouver p_lim -------------------------------------------------
Nous cherchons maintenant à déterminer $p_{lim}$.
La condition limite naturelle est celle où la valeur propre maximale est égale au bord droit du support de la mesure spectrale de la matrice A.
On a alors 
\begin{align*}
	&\Leftrightarrow \lambda^+ = z1\\
	&\Leftrightarrow 2 \sigma = \frac{c_{in} - c_{out}}{q\sqrt{n}} + \frac{q\sqrt{n}\sigma^2}{c_{in} - c_{out}}\\
	&\Leftrightarrow 0 = \beta \sigma^2 - 2 \sigma + \alpha \\
	&\text{après résolution de l'équation on obtient une unique solution}\\
	&\Leftrightarrow p_{in} - p_{out} = \frac{q\sigma}{\sqrt{n}}  \\
\end{align*}
Donc
\begin{equation}
	p_{lim} = \frac{\sqrt{q(\sigma_{in}^2 + (q-1)\sigma_{out}^2)}}{\sqrt{n}} = \frac{q\sigma}{\sqrt{n}}  \\
\end{equation}


% ------------------------------------------------- 5- résumer -------------------------------------------------
Ci-dessous le bilan de la généralisation:
\begin{align*}
	\sigma^2&: \frac{\sigma_{in}^2 + (q-1)\sigma_{out}^2}{q} \\
	z1&: \frac{c_{in} - c_{out}}{q\sqrt{n}} + \frac{q\sqrt{n}\sigma^2}{c_{in} - c_{out}}\\
	z2&: \frac{c_{in} + (q-1)c_{out}}{q\sqrt{n}} + \frac{q\sqrt{n}\sigma^2}{c_{in} + (q-1)c_{out}}\\
	p_{lim}&: \frac{\sqrt{q(\sigma_{in}^2 + (q-1)\sigma_{out}^2)}}{\sqrt{n}} \\
\end{align*}
% ------------------------------------------------- 6- Simulation -------------------------------------------------

\begin{figure}[H]
\centering
\includegraphics[scale=0.6]{static/graph_q7_n100_pin08_pout0011.png.png}
\caption{Graphe généré à partir des paramètres: $q=7$ $n=100$, $p_{in}=0.8$, $p_{out}=0.01$}
\end{figure}
\nocite{*}
\bibliographystyle{plain}
\bibliography{b}
\end{document}
