\documentclass{article} % article, report, book, slides, beamer, lettre, memoir
 
% =============================
% Import
\usepackage[francais]{babel}
\usepackage[T1]{fontenc}
\usepackage{fontspec} 
\newcommand\fr{\selectlanguage{francais}}
\usepackage{titlesec}
\usepackage{graphicx}
\usepackage{float}
\usepackage{amstext}
\usepackage{amssymb}
\usepackage{subcaption}
\usepackage{geometry}
\usepackage[colorlinks]{hyperref}
\usepackage{amsmath}
\usepackage{bbm}
\usepackage{blkarray}
\usepackage{empheq}

\newlength\dlf  % Define a new measure, dlf
\newcommand\alignedbox[2]{& {\settowidth\dlf{$\displaystyle #1$}\addtolength\dlf{\fboxsep+\fboxrule}\hspace{-\dlf}\boxed{#1 #2}}}
\geometry{
 a4paper,
 total={170mm,257mm},
 left=20mm,
 top=20mm,
 }


\newcommand\numberthis{\addtocounter{equation}{1}\tag{\theequation}}

\graphicspath{ {./static/} }

\newtheorem{theorem}{Theorem}
\newtheorem{mydef}{Definition}

\title{Graph spectra and the detectability of community structure in networks}
\author{Hugues-Vincent Ropert}
\date{Mai 2018}

% \newcommand{\example}{\textit{example} }

% \titleformat{\chapter}
%   {\Large\bfseries} % format
%   {}                % label
%   {0pt}             % sep
%   {\huge}           % before-code

\titleformat{\section}[block]
  {\fontsize{12}{15}\bfseries\sffamily}
  {\thesection}
  {1em}
  {}
\titleformat{\subsection}[block]
  {\Large\bfseries}
  {\thesubsection}
  {1em}
  {}


\begin{document}
\maketitle

\tableofcontents
% \newpage

\part{Vue d'ensemble de la théorie du clustering spectral de graphes}
\section{La détection de communauté}
\subsection{Motivations}
La théorie des réseaux, a pour but d'analyser les graphes correspondant à des réseaux tels que celui de l'Internet, de la politique ou de la classification en biologie.
Chacun de ces graphes ont des propriétés spécifiques telles que:
\begin{itemize}
 	\item[-] Centrality ;
 	\item[-] ``small world effect'' ;
 	\item[-] Clustering ;
 	\item[-] Efficiency ;
 	\item[-] Degree distribution ; 
 	\item[-] Community Structure.\\
 \end{itemize}
 C'est cette dernière propriété qui va nous intéresser, à savoir la structure de communauté d'un graphe.
 L'idée de communauté correspond à l'intuition selon laquelle il y a des nœuds qui ont un lien étroit et forment des sous ensembles de nœuds de ce graphe.
 Par exemple dans le graphe des représentants politique Français, les hommes politiques d'un même partie appartiendraient à une même communauté.
 Cependant, dans cet exemple les choses peuvent être plus subtiles que ça, c'est là qu'intervient la détection de communauté.

\subsection{Méthodes}
Il y a différentes classes de méthodes pour la détection de communauté, en voici une liste non exhaustive:
\begin{itemize}
	\item[-] Graph partitioning
	\item[-] Hierarchical clustering
	\item[-] Partitional clustering
	\item[-] Spectral clustering \\
\end{itemize}
\subsection{Perspectives}
Les limites des modèles spectraux qui sont en développement pour la détection de communauté sont les suivantes:
\begin{itemize}
	\item[1-] Seuil à partir duquel la méthode spectrale n'est plus en mesure de détecter le communauté ;  
	\item[2-] Temps de calcul sur de grands graphes;  
	\item[3-] Raffinement des modèles de communautés.  
\end{itemize}

\subsection{Comparaison des performances}
La notion de structure de communauté n'a pas de définition consensuelle.
Ceci est également vrai pour les métriques qui mesurent les performances des différentes méthode.
Ci-dessous une liste des métriques qui seront utilisées ultérieurement.
Pour une liste complète voir \cite[p.77-79]{Community_detection_in_graphs}.
\begin{itemize}
	\item[-] Fraction of correctly classified vertices ; 
	\item[-] Fowlkes and Mallows metric; 
	\item[-] Rand Index; 
	\item[-] Normalized Mutal Information; 
\end{itemize}
\section{Spectral clustering}
L'idée centrale des techniques de clustering spectral est qu'un graphe est représentable par une matrice à partir de laquelle on peut utiliser les techniques d'analyse de l'algèbre linéaire.

Un graphe $G$ est la donnée d'un couple $(V, E)$ tel que $V$ est une ensemble de nœuds et $E$ et un ensemble d'arêtes (i.e un couple $(i,j)$ où $i,j \in V$).
A partir de cette définition on peut représenter le graphe par une matrice dont les éléments correspondent à certaines données de $G$.
Il existe tout un ensemble de matrices représentant le graphe:
\begin{itemize}
  	\item[-]  Adjencency Matrix $= A$;
  	\item[-]  Laplacian Matrix $= L$;
  	\item[-]  Modularity Matrix $= M$;
  	\item[-]  Bethe Hessian Matrix $= H$;
  	\item[-]  "$\alpha$-normalized” Adjacency Matrix $= D$.\\
\end{itemize}
Par exemple, la matrice d'adjacence d'un graphe, notée $A$, est définie telle que $\forall i,j \in V, \; A_{ij} = \mathbbm{1}_{(i,j) \in E}$.
Une colonne $i$ représente le nœud $i$ dont chaque composante, $j$, est égale à $1$ si il existe une arête entre $i$ et $j$ et $0$ sinon.\\

Ci-dessous un graphe qui permettant d'avoir une vue d'ensemble sur l'avancement des méthodes de spectrales de détection de communautés.
Cette liste n'est pas exhaustive.
\begin{figure}[H]
\centering
\includegraphics[scale=0.5]{static/graph_research.png}
\caption{graphe des différentes méthodes de ``graph spectral clustering''}
\end{figure}

\subsection{Algorithme de clustering spectral de graphe}
 \label{par:algo spectral clustering}
La procédure pour associer une communauté aux nœuds de $G$ via une méthode spectrale est la suivante: 
\begin{itemize}
	\item[1-] Calcul des vecteurs propres, $v_i$, d'une matrice représentant le Graphe $G$ ;
	\item[2-] Sélection des $l$ vecteurs propres portant l'information de la structure de communauté ; 
	\item[3-] Construction de la matrice $W = [v_1, \cdots, v_l] \in \mathbb{R}^{n\times l}$ ; 
	\item[4-] Projection des vecteur lignes $r_j$ de $W$ sur l'espace de dimension $l$, où chaque $r_j$ correspond  au nœud $j$ ; 
	\item[5-] Catégorisation des vecteurs $r_j $ dans une communauté via des algorithmes de clustering: \textit{K-Means}, \textit{Expetation-Maximization}, \textit{Support vector machine }, etc.\\
\end{itemize}

\par{\underline{Étape 1:}}
La matrice $A$ est carrée, par conséquent elle peut être interprétée comme la représentation d'un endomorphisme dans un espace X de dimension $n$ (nombre de nœuds dans $G$).
Soit la base canonique $(e_i)_{i=1:n}$, chaque $e_i$ correspond au nœud $i$.
Les vecteurs propres de $A$ sont donc des combinaisons linéaires des nœuds de $G$.
Les nœuds concernés ont donc une certaine dépendance.
\par{\underline{Étape 2:}}
Les entrées de la Matrice $A$ sont modélisées par des variables aléatoires vérifiant certaines hypothèses (e.g centrées, moments finis, indépendantes ...).
La théorie des matrices aléatoires nous dit qu’asymptotiquement la mesure spectrale de A converge vers une loi déterministe $\mu$ (loi qui dépend de la matrice étudiée).
Les valeurs propres de $A$ distribuées selon $\mu$ peuvent être interprétées comme du bruit lié aux fluctuations aléatoires des simulations.
Elles nous renseigne en rien sur la structure non aléatoire des entrées de $A$ .\\
Cependant, lorsque le graphe admet une structure de communauté, les entrées ne sont plus tout à fait indépendantes.
Dans ce cas de figure, la théorie prévoit que des valeurs propres sortent du support de la mesure spectrale $\mu$.
Ce sont ces valeurs propres qui contiennent l'information de la structure de communauté de $G$.\\
Autrement dit, si la structure de communauté de $G$ n'est pas assez ``explicite'', les valeurs propres censées porter l'information des communautés ne sortiront pas du support de la mesure spectrale théorique, et donc, seront interprétées comme du simple bruit.
Dans cette situation, la méthode spectrale est incapable de déceler la structure de communauté de $G$.
\par{\underline{Étape 4:}}
Les vecteurs colonnes de W correspondent à des combinaisons linéaires de nœuds.
De par la construction de A, les entrées $i$ de ces vecteurs colonnes sont la somme des arêtes entre les nœuds sous-jacent et le nœud $i$.
Les vecteurs propres que l'on a gardés sont ceux qui portent l'information des communautés.
Donc chaque vecteur ligne représente un nœud du graphe et l'espace dans lequel il se meut constitue l'information de la structure de communauté.\\

Les algorithmes de détection de communauté spectraux à partir des opérateurs linéaires tels que la matrice ``non-backtracking'' et la matrice ``Bethe Hessian'' sont similaires et sont décrits dans l'article \cite[Spectral detection in the censored block model]{Spectral_Detection_in_the_Censored_Block_Model}.

\subsection{Non-backtracking matrix}
\subsubsection{Localization and Centrality in Networks }
\subsubsection{Spectral Redemption: Clustering Sparse Networks}
\subsubsection{Percolation on Sparse Networks}


\part{Article: Graph spectra and the detectability of community structure in networks - Raj Rao Nadakuditi and M. E. J. Newman}
% \part{partie}
% \chapter{chapitre}
% \section{section}
% \subsection{sous-section}
% \subsubsection{sous-sous-section}
% \paragraph{paragraphe}
% \subparagraph{sous-paragraphe}
\section{Théorie}
\subsection{Contexte: Stochastic Block Model à 2 communautés}
Le but de ce papier est d'expliquer en détail la méthode de détection de communauté via la théorie spectrale de l'article de RR. Nadakuditi et M. E. J. Newmann.
C'est un article qui décrit les limites d'un modèle spectral dans un cadre spécifique à deux communautés.

Nous allons nous placer dans le contexte d'un Stochastic Block Model (SBM) à deux communautés (i.e \textbf{q}=2).
C'est un graphe \textbf{G} non orienté à \textbf{n} nœuds dont chaque arête entre deux nœuds suit une loi de Bernoulli de paramètre $p_{in}$ si les nœuds appartiennent à la même communauté et $p_{out}$ si les nœuds ne sont pas dans la même communauté.  

Nous noterons \textbf{A} la matrice d'adjacence du graphe \textbf{G}.
Elle est symétrique de par le fait que le graphe soit non orienté.
Nous supposerons d'ailleurs, que ses éléments sont rangés dans l'ordre de leur communauté.
Dans notre cas avec \textbf{q}=2, les $\mathbf{n}/2$ premières lignes correspondent aux noeuds de la communauté 1, et les $\mathbf{n}/2$ dernières la communauté 2.
Même chose pour les colonnes, par symétrie de \textbf{A}.\\
Cette disposition des éléments ne change pas le résultat final de l'analayse.
En effet, on sait que toutes les matrices congruentes représentes la même forme bilinéaire dans des bases différentes. 
Or le but de la procédure qui va suivre va être d'analyser la distribution des valeurs propres de notre matrice.
Par conséquent ranger les colonnes d'une matrice selon un ordre arbitraire correspond à la même forme bilinéaire et donc aux mêmes valeurs propres.\\

Chaque élément de la matrice \textbf{A} est simulé par une loi de Bernoulli avec: 
\begin{equation}\label{rq:probability}
 A_{ij} \sim \left\{
  \begin{array}{lr}
    B(p_{in}) & : (i,j < \frac{n}{2}) \; ou \; (i,j \ge \frac{n}{2}) \\
    B(p_{out}) & : else \; where
  \end{array}
\right.\nonumber
\end{equation}
\begin{equation} 
A_{ij} = A_{ji}\nonumber
\end{equation}


\subsection{Analyse spectrale de la matrice d'adjacence \textbf{A}}\label{ch:Analyse spectrale de la matrice d'adjacence}
L'idée générale de l'analyse spectrale qui va suivre est de nous ramener à un régime spectral de grande matrices aléatoires connu. 
Dans notre cas, nous verrons que le régime associé à notre matrice d'adjacence (SBM $\textbf{q}=2$) est celui du théorème de \textit{Wigner} avec perturbation de rang fini.
La trame sera la suivante:
\begin{itemize}
 	\item[1-] réécriture de la matrice $A = \langle A \rangle + X$;
 	\item[2-] étude de la mesure spectrale de X;
 	\item[3-] étude de la mesure spectrale de B où $B = X + P_1$;
 	\item[4-] étude de la mesure spectrale de $A = B + P_2 = X + P_1 + P_2$.
 \end{itemize} 
où $P_1$, $P_2$ sont des perturbations de rang 1 et $\langle A \rangle$ correspond à la moyenne de A du SBM.\\
% ------------------------------------------------- equation (1) -------------------------------------------------
\subsubsection*{1- Équation de $\langle A \rangle$}
En partant de $A$, on définit $\langle A \rangle$ comme la matrice dont les entrées $\langle A \rangle _{ij} = \mathbb{E}(A_{ij})$
Soit
\begin{align*}
	\langle A \rangle 
	&=
	\begin{bmatrix}
		p_{in} & p_{out}\\
		p_{out} & p_{in}\\
	\end{bmatrix} \otimes \mathbf{1}_{\frac{n}{2}}\mathbf{1}_{\frac{n}{2}}^T \times \frac{n}{2}\\
	&=
	\begin{bmatrix}
		c_{in} & c_{out}\\
		c_{out} & c_{in}\\
	\end{bmatrix} \otimes \mathbf{1}_{\frac{n}{2}}\mathbf{1}_{\frac{n}{2}}^T \times \frac{1}{2}\\
	&=\left(\frac{1}{2} \times
	\begin{bmatrix}
		c_{in} + c_{out} & c_{in} + c_{out}\\
		c_{in} + c_{out} & c_{in} + c_{out}\\
	\end{bmatrix} 
	+ \frac{1}{2} \times
	\begin{bmatrix}
		c_{in} - c_{out} & -c_{in} + c_{out}\\
		-c_{in} + c_{out} & c_{in} - c_{out}\\
	\end{bmatrix}
	\right) \otimes \mathbf{1}_{\frac{n}{2}}\mathbf{1}_{\frac{n}{2}}^T \times \frac{1}{2}\\
	&=\left(\frac{1}{2} (c_{in} + c_{out})\times \mathbf{1}_2\mathbf{1}_2^T
	+ \frac{1}{2} (c_{in} - c_{out}) \times \mathbf{u}_2\mathbf{u}_2^T
	\right) \otimes \mathbf{1}_{\frac{n}{2}}\mathbf{1}_{\frac{n}{2}}^T\\
	&= \frac{1}{2}(c_{in} + c_{out})\mathbf{1}_n\mathbf{1}_n^T + \frac{1}{2}(c_{in} - c_{out})\mathbf{u}_n\mathbf{u}_n^T \numberthis \label{eq:1}
\end{align*}
Avec
\begin{align*}
c_{in} &= np_{int} \\
c_{out} &= np_{out}\\
\mathbf{1}_n &= (\underbrace{1,\ldots,1}_{n\text{-times}})/\sqrt{n}\\
\mathbf{u}_n &= (\underbrace{1,\ldots,1}_{\frac{n}{2}\text{-times}}, \underbrace{-1,\ldots,1}_{\frac{n}{2}\text{-times}})/\sqrt{n}\\
\langle\mathbf{1_n}|\mathbf{u_n}\rangle &= 0
\end{align*}


À présent \textbf{A} peut être écrite sous la forme $A = \langle A \rangle + X$.
La matrice X est interprétable comme la déviation entre la matrice d'adjacence du graphe et sa moyenne.
La matrice X est par définition une matrice aléatoire symétrique à entrées indépendantes et de moyenne 0.
Essayons d'analyser sa mesure spectrale.
On a 

\begin{equation} \label{eq: X}
X = A - \langle A \rangle
\end{equation}
\begin{equation}
	X_{ij}  = \left\{
	\begin{array}{lr}
		B_{ij}(p_{in}) - p_{in} & : (i,j < \frac{n}{2}) \; ou \; (i,j \ge \frac{n}{2}) \\
		B_{ij}(p_{out}) - p_{out} & : else \; where
	\end{array}
\right.\nonumber
\end{equation}
Où $B_{ij}(p) \sim B(p)$, $B(p)$ loi de Bernoulli de paramètre p et $B_{ij} = B_{ji}$\\


% ------------------------------------------------- 2- trouver X  -------------------------------------------------
\subsubsection*{2- Recherche de la mesure spectrale de $\frac{X}{\sqrt{n}}$}
Nous aimerions modifier la forme des entrées $X_{ij}$ en $\sigma_{ij} Z_{ij}$, où $Z_{ij}$ est une variable aléatoire centrée réduite, afin de nous ramener à des théorèmes connus.

Pour $(i,j < \frac{n}{2}) \; ou \; (i,j \ge \frac{n}{2}) $ on a:
\begin{align*}
\mathbb{E}(X_{ij}) &= \mathbb{E}(B(p_{in}))- p_{in} = 0\\
\sigma_{in}^2 &= \mathbb{V}(X_{ij}) \\ 
			  &= \mathbb{V}(B(p_{in})) \\
			  &= p_{in} (1 - p_{in})
\end{align*}
même raisonnement avec $p_{out}$ 
\begin{align*}
\sigma_{out}^2 =  p_{out} (1 - p_{out})
\end{align*}
finalement on obtient 
\begin{equation}
	X_{ij} \sim \left\{
	\begin{array}{lr}
		\sigma_{in} Z_{ij} & : (i,j < \frac{n}{2}) \; ou \; (i,j \ge \frac{n}{2}) \\
		\sigma_{out} Z_{ij} & : else \; where
	\end{array}
\right.\nonumber
\end{equation}
Où $Z_{ij} = \frac{B(p) - p}{\sqrt{p(1-p)}} \;\;avec \; p = p_{in} \; ou \; p_{out}$\\

% ------------------------------------------------- théorème 1 -------------------------------------------------
Le théorème de Wigner ne fonctionne que pour les matrices à entrées iid. 
Il existe un théorème lorsque le profil de variance de la matrice aléatoire étudié est stochastique.
\begin{theorem}\label{th:1}
Soit $W \in M_{n}(\mathbb{R})$, $W_{ij}$ sont des variables aléatoires réelles\\
On suppose que les $(W_{ij})_{i \leq j}$ sont indépendantes telles que $\mathbb{E}(W_{ij}) = 0$, $\sigma_{ij}^2 = \mathbb{E}(W_{ij}^2) < \infty$, $W_{ij} = W_{ji}$\\
On appelle profil de variance la matrice symétrique
\begin{align*}
	\tilde V_n = (\sigma_{ij}^2)_{1 \leq i,j \leq n}
\end{align*}
et profil de variance normalisé 
\begin{align*}
	V_n = \left(\frac{\sigma_{ij}^2}{n}\right)_{1 \leq i,j \leq n}
\end{align*}
On suppose que $V_n$ est stochastique: $\forall i = 1:n , \; \sum_{j=1}^{n}V_{ij} = \sigma^2$\\
Alors la mesure spectrale $L_{n}$ de $\frac{W}{\sqrt{n}}$ vérifie

\begin{equation}
	L_n\xrightarrow[n \to +\infty]{etr} \mathbb{P}_{wig}\nonumber \; p.s
\end{equation}\\
où $\mathbb{P}_{wig}$ est la loi de Wigner de densité $f_{wig}(x)= \frac{\sqrt{(4\sigma^2 - x^2)_+}}{2\pi\sigma^2}$
\end{theorem}

Soit $V$ le profil de variance de $\frac{X}{\sqrt{n}}$. $\forall i = 1:n$ on a: 
\begin{align*} 
\sigma_{i \cdot}^2 &= \sum_{j=1}^{n}V_{ij}  \\
 		&= \boxed{\frac{\sigma_{in}^2 + \sigma_{out}^2}{2} = \sigma^2}
\end{align*}
Si on note $\rho(x)$ la densité spectrale de $\frac{X}{\sqrt{n}}$ on a;
\begin{equation}
	\rho(x) = \frac{\sqrt{(2(\sigma_{in}^2 + \sigma_{out}^2) - x^2)_+}}{\pi(\sigma_{in}^2 + \sigma_{out}^2)}
\end{equation}

% ------------------------------------------------- 3- étude de B  -------------------------------------------------
\subsubsection*{3- Étude de la mesure spectrale de $\frac{X}{\sqrt{n}}$ et recherche de ces valeurs propres maximales}
Nous venons de trouver la densité spectrale de $\frac{X}{\sqrt{n}}$.
Nous voulons à présent étudier ces valeurs propres.
Pour ce faire nous allons séparer l'équation $\frac{X}{\sqrt{n}} = \frac{1}{\sqrt{n}}(A - \langle A \rangle)$ en deux parties, à savoir $\frac{X}{\sqrt{n}} = \frac{1}{\sqrt{n}}(B \;+\; \frac{1}{2}(c_{in} + c_{out})\mathbf{11}^T)$\\
Dans la suite de l'étude nous noterons B la matrice de modularité telle que 
\begin{align*} 
B :&= X + \frac{1}{2}(c_{in} - c_{out})\mathbf{uu}^T \\
\frac{B}{\sqrt{n}} &= \frac{X}{\sqrt{n}} + \frac{1}{2\sqrt{n}}(c_{in} - c_{out})\mathbf{uu}^T\\
\end{align*}
Le raisonnement qui va suivre est une heuristique. En effet, pour compléter rigoureusement la preuve qui va suivre, il faudrait vérifier la convergence du terme $\frac{B}{\sqrt{n}}$ en entier.\\

Soit v le vecteur propre de $\frac{B}{\sqrt{n}}$ associé à $\lambda_{max}$ et soit $\Gamma = \frac{X}{\sqrt{n}}$

\begin{align} 
\frac{B}{\sqrt{n}}\mathbf{v} &= \lambda_{max}\mathbf{v} \nonumber\\
(\Gamma - \lambda_{max}I)\mathbf{v} &= -\frac{1}{2\sqrt{n}}(c_{in} - c_{out})\mathbf{uu}^T \mathbf{v} \nonumber\\
 \mathbf{u^Tv} &= -\frac{1}{2\sqrt{n}}(c_{in} - c_{out})\mathbf{u^T}(\Gamma - \lambda_{max}I)^{-1}\mathbf{uu}^T \mathbf{v} \nonumber\\
 1 &= -\frac{1}{2\sqrt{n}}(c_{in} - c_{out})\mathbf{u^T}(\Gamma - \lambda_{max}I)^{-1}\mathbf{u} \label{eq:3}
\end{align}

% ------------------------------------------------- Théorème de Wigner isotrope -------------------------------------------------
\begin{theorem}[Théorème de Wigner isotrope]\label{th:2}

Soit $W \in M_{n}(\mathbb{R})$, telle que $W_{ij}$ sont des variables aléatoires réels, $\mathbb{E}(W_{ij}) = 0$, $\mathbb{E}(W_{ij}^2) < \infty$, $W_{ij} = W_{ji}$, W à un profil de variance V tel que $\forall i = 1:n , \; \sum_{j=1}^{n}V_{ij} = \sigma^2$ pour $\sigma \in \mathbb{R}$\\
Soit $Q(z)$ la résolvante de W
\begin{align*} 
Q(z) = (W - zI)^{-1}\\
\end{align*}
Soient $\mathbf{u}$, $\mathbf{v}$ des vecteurs déterministes tels que $\|\mathbf{u}\|, \|\mathbf{v}\| < \infty$.\\
Alors 
\begin{align*} 
\mathbf{u}^*Q(z)\mathbf{v} - \langle \mathbf{u}, \mathbf{v} \rangle g_{wig}^{\sigma^2}(z) \xrightarrow[n \to +\infty]{} 0\\
\end{align*}
où $g_{wig}^{\sigma^2}$ est la transformé de Stieltjes de la loi de Wigner de paramètre $\sigma^2$ qui satisfait l'équation
\begin{equation}
	\sigma^2g_{wig}^{\sigma^2}(z)^2+z g_{wig}^{\sigma^2}(z)+1=0
\end{equation}\\
\end{theorem}

Reprenons l’équation \eqref{eq:3}, en appliquant le Théorème de Wigner isotrope et en s'assurant que tous les termes de droite convergent, nous avons:
\begin{align*}
\mathbf{u^T}(\Gamma - \lambda_{max}I)^{-1}\mathbf{u} \xrightarrow[n \to +\infty]{} g_{wig}^{\sigma^2}(\lambda_{max})
\end{align*}
or $g_{wig}^{\sigma^2}(\lambda_{max})$ satisfait l'équation suivante 
\begin{align}
	\sigma^2g_{wig}^{\sigma^2}(\lambda_{max})^2+\lambda_{max}g_{wig}^{\sigma^2}(\lambda_{max})+1=0 \implies g_{wig}^{\sigma^2}(\lambda_{max}) = \frac{- \lambda_{max} \pm \sqrt{(\lambda_{max}^2 - 4\sigma^2)}}{2\sigma^2}
\end{align}
Donc
\begin{align}
	\eqref{eq:3} &\Leftrightarrow1 = -\frac{1}{2\sqrt{n}}(c_{in} - c_{out}) g_{wig}^{\sigma^2}(\lambda_{max}) \nonumber\\
	&\Leftrightarrow 1 = -\frac{1}{2\sqrt{n}}(c_{in} - c_{out}) \frac{- \lambda_{max} - \sqrt{(\lambda_{max}^2 - 4\sigma^2)}}{2\sigma^2} \nonumber\\
	&\Leftrightarrow \boxed{\lambda_{max}=\frac{(c_{in} - c_{out})}{2\sqrt{n}} + \sqrt{n}\frac{\sigma_{in}^2 + \sigma_{out}^2}{c_{in} - c_{out}}} \label{z_1}
\end{align}

de plus $\frac{1}{2\sqrt{n}}(c_{in} - c_{out}) = \mathcal{O}(\sqrt{n})$ et $\sqrt{n}\frac{\sigma_{in}^2 + \sigma_{out}^2}{c_{in} - c_{out}} =  \mathcal{O}\left(\frac{1}{\sqrt{n}}\right)$\\
 
Comme $A = X + \frac{1}{2}(c_{in} + c_{out})\mathbf{11}^T + \frac{1}{2}(c_{in} - c_{out})\mathbf{uu}^T$ et $\langle\mathbf{u}, \mathbf{1}\rangle = 0 \; \|\mathbf{u}\|=\|\mathbf{v}\|=1$ on a
\begin{align}
	&\Leftrightarrow (\Gamma + \alpha\mathbf{11}^T + \beta \mathbf{uu}^T)\mathbf{v} = \lambda \mathbf{v}\nonumber\\
	&\Leftrightarrow (\Gamma - \lambda I)\mathbf{v} = -\alpha\mathbf{11}^T\mathbf{v} - \beta \mathbf{uu}^T\mathbf{v} \nonumber\\
	&\Leftrightarrow \mathbf{1^Tv} = -\alpha\mathbf{1^T} (\Gamma - \lambda I)^{-1}\mathbf{11}^T\mathbf{v} - \beta \mathbf{1^T} (\Gamma - \lambda I)^{-1}\mathbf{uu}^T\mathbf{v} \nonumber\\
	&\xrightarrow[n \to +\infty]{} 1 = -\alpha g_{wig}^{\sigma^2}(\lambda) \nonumber\\
	&\Leftrightarrow 1 = -\alpha  \frac{- \lambda - \sqrt{(\lambda^2 - 4\sigma^2)}}{2\sigma^2}\nonumber\\
	&\Leftrightarrow \boxed{\lambda = \frac{(c_{in} + c_{out})}{2\sqrt{n}} + \sqrt{n}\frac{\sigma_{in}^2 + \sigma_{out}^2}{c_{in} + c_{out}}} \label{z_2}
\end{align} 

On a finalement une matrice A qui est la somme d'une matrice aléatoire X et de deux perturbations de rang 1.
De fait, la mesure spectrale de A à deux valeurs propres qui sortent du support de la distribution de Wigner, à savoir $\lambda_{max}$ et $\lambda$.
Par la suite nous noterons $z_1 = \lambda_{max}$ et $z_2 = \lambda$, et nous appellerons ``bulk'' le support de la mesure spectrale $\rho(z)$.

\subsection{Interprétation des résultats obtenus}
\label{subsec:1.3}
Si les 2 plus petites valeurs propres sont inférieures au bord gauche du bulk (ici $\lambda^- = -2\sigma$) alors le graphe admet une structure de communauté ``disassortative''.
Si les 2 plus grandes valeurs propres sont supérieures au bord droit du bulk (ici $\lambda^+ = 2\sigma$) alors le graphe admet une structure de communauté ``assortative''.
Si $0$ ou $1$ valeur propre sort du bulk alors la méthode spectrale ne peut rien conclure sur la structure de communauté du graphe G.\\

Il est important de remarquer que $z_2$ est toujours supérieure au bord droit du bulk, indépendamment des valeurs de $p_{in}$ et $p_{out}$.
Par conséquent, c'est $z_1$ qui nous indique si oui ou non il y a une structure de communauté dans le graphe.\\

Nous cherchons à déterminer $p_{lim} = p_{in} - p_{out}$, qui est la condition limite supérieure sur les paramètres du SBM pour que l'algorithme puisse détecter la structure de communauté ``assortative'' du graphe G.
Dans la mesure où  $\lambda^+$ est positif ou nul, lorsque $p_{in} - p_{out} \in [0, p_{lim}]$ alors la méthode spectrale est incapable de conclure sur la structure ``assortative''.\\
 
La condition limite naturelle est celle où la valeur propre maximale est égale au bord droit du support de la mesure spectrale de la matrice A.
À partir de cette condition nous allons essayer de retrouver $p_{lim}$.\\
On a alors 
\begin{align*}
	&\Leftrightarrow \lambda^+ = \lambda_{max}\\
	&\Leftrightarrow \sqrt{2(\sigma_{in}^2 + \sigma_{out}^2)} = \frac{(c_{in} - c_{out})}{2\sqrt{n}} + \sqrt{n}\frac{\sigma_{in}^2 + \sigma_{out}^2}{c_{in} - c_{out}}\\
	&\Leftrightarrow \sqrt{2S} = \alpha + \beta S &\; avec  \; S = \sigma_{in}^2 + \sigma_{out}^2,\;\alpha=\frac{(c_{in} - c_{out})}{2\sqrt{n}}, \; \beta=  \frac{\sqrt{n}}{c_{in} - c_{out}}\\
	&\Leftrightarrow 2S = \alpha^2 + 2\alpha \beta S +\beta^2 S^2\\
	&\Leftrightarrow 0 = \beta^2 S^2 + 2(\alpha \beta - 1)S+ \alpha^2 \\
	&\Leftrightarrow p_{in} - p_{out} = \frac{\sqrt{2(\sigma_{in}^2 + \sigma_{out}^2)}}{\sqrt{n}}  \\
\end{align*}
Donc
\begin{equation}
	\boxed{p_{lim} = \frac{\sqrt{2(\sigma_{in}^2 + \sigma_{out}^2)}}{\sqrt{n}}= \frac{2\sigma}{\sqrt{n}} = \mathcal{O}\left(\frac{1}{\sqrt{n}}\right)}\\
\end{equation}

\paragraph{}\label{rq:ngrand}
D'après ce modèle, plus on a de noeuds dans le graphe, plus on a de l'information, et donc moins on a de chance de tomber sur cet intervalle d’indécidabilité.\\

\subsection{Bilan}
Ci-dessous un tableau comparatif entre les résultats obtenus et ceux de l'article \cite{raj_rao}.
L'essence de la divergence intervient lorsque qu'il est défini que $\mathbb{V}(X_{ij}) = \frac{(c_{in} + c_{out})}{2n}$.
Nous avons normalisé les équations de l'article original par $\frac{1}{\sqrt{n}}$ afin de pouvoir les comparer avec nos résultats.\\
\underline{Rappel}: $\sigma_{in}^2=p_{in}(1-p_{in})$ ; $\sigma_{out}^2=p_{out}(1-p_{out})$ ; $c_{in} = np_{in}$ ; $c_{out} = p_{out}$

\renewcommand{\arraystretch}{2}
\begin{table}[h] 
	\label{tab:bilan}
	\centering
    \begin{tabular}{|l|l|l|}
    \hline
      &  Résultats de l'article  &  Résultats obtenus \\ \hline
    $\rho (z)$  &  $\frac{1}{\pi\sqrt{n}}\frac{\sqrt{(2(c_{in} + c_{out}) - x^2)_+}}{(c_{in} + c_{out})}$&  $\frac{\sqrt{(2(\sigma_{in}^2 + \sigma_{out}^2) - x^2)_+}}{\pi(\sigma_{in}^2 + \sigma_{out}^2)}$\\ \hline
    $z_1$&  $\frac{c_{in}-c_{out}}{2\sqrt{n}} + \sqrt{n}\frac{c_{in}+c_{out}}{c_{in}-c_{out}}$ &  $\frac{c_{in}-c_{out}}{2\sqrt{n}} + \sqrt{n}\frac{\sigma_{in}^2+\sigma_{out}^2}{c_{in}-c_{out}}$\\ \hline
    $z_2$&  $\frac{c_{in}+c_{out}}{2\sqrt{n}} + \sqrt{n}$ &  $\frac{c_{in}+c_{out}}{2\sqrt{n}} + \sqrt{n}\frac{\sigma_{in}^2+\sigma_{out}^2}{c_{in}+c_{out}}$\\ \hline
    $p_{lim}$&  $\frac{\sqrt{2(c_{in} - c_{out})}}{n}$ &  $\frac{\sqrt{2(\sigma_{in}^2 + \sigma_{out}^2)}}{\sqrt{n}}$\\ \hline
    \end{tabular}
\end{table}

Les résultats quantitatifs empiriques que nous avons obtenu sont différents de ceux de l'article de référence.
Au contraire, les simulations corroborent les formules trouvées ci-dessus.
Nous le verrons plus précisément dans la section suivante.   

\section{Simulation}
Le but est de générer une matrice d'adjacence A sous les mêmes hypothèses que \eqref{eq:1} en faisant varier les différents paramètres, à savoir: $n$, $p_{in}$, $p_{out}$ \\

\begin{figure}[h]
\centering
\includegraphics[scale=0.6]{static/graph_n50_pin08_pout01.png}
\caption{Graphe généré à partir des paramètres: $n=50$, $p_{in}=0.8$, $p_{out}=0.1$}
\end{figure}

L'élément discriminant du test spectral étant la variable $ \Delta p= p_{in} - p_{out}$, nous allons tester 3 valeurs représentatives des différents types de résultats:
\begin{itemize}
	\item[1-] $\Delta p \in [-1,\: 0] \implies$ le graphe ne comporte pas de structure de communauté ;
	\item[2-] $\Delta p \in [0,\: p_{lim}] \implies$ le graphe comporte une structure de communauté mais la méthode spectrale ne réussi pas à la trouver ;
	\item[2-] $\Delta p \in [p_{lim},\: 1] \implies$ le graphe comporte une structure de communauté.\\
\end{itemize}

Par la suite nous allons tester pour les valeurs $\Delta p= 0.5, 0.08, -0.5$ .
De plus nous ferons une première vague de test avec $n=100$ et une autre avec $n=1000$.
La terminologie utilisée dans les figures ci-dessous est:
\begin{itemize}
	\item[- \underline{$n,\: p_{in},\: p_{out},\: p_{lim},\: z1,\: z2$}:] sont identiques aux notations utilisées jusqu'à présent ;
	\item[- \underline{$z1\: theoric, \:z2\: theoric$}:] sont les plus grandes valeurs propres $z1$ et $z2$ calculées via les équations \eqref{z1} et \eqref{z2} ;    
	\item[- \underline{$p_{in}\: estimated, \:p_{out}\: estimated$}:] sont les probabilités du SBM calculées a posteriori grâce aux valeurs propres $z1$ et $z2$.\\
\end{itemize}
Les $p_{in}\: estimated, \:p_{out}\: estimated$ sont calculés via un calcul d'optimisation (la fonction ``fsolve'' en python).
\begin{figure}[H]
	\begin{subfigure}{.5\textwidth}
		\centering
		\includegraphics[scale=0.58]{static/spectral_n100_pin08_pout03.png}
		\caption{$n=100$, $\Delta p=0.5$}
	\end{subfigure}
	\begin{subfigure}{.5\textwidth}
		\centering
		\includegraphics[scale=0.58]{static/spectral_n100_pin05_pout042.png}
		\caption{$n=100$, $\Delta p=0.08$}
	\end{subfigure}
	\begin{subfigure}{.5\textwidth}
		\centering
		\includegraphics[scale=0.58]{static/spectral_n100_pin02_pout07.png}
		\caption{$n=100$, $\Delta p=-0.5$}
	\end{subfigure}
	\begin{subfigure}{.5\textwidth}
		\centering
		\includegraphics[scale=0.58]{static/spectral_n1000_pin08_pout03.png}
		\caption{$n=1000$, $\Delta p=0.5$}
	\end{subfigure}
	\begin{subfigure}{.5\textwidth}
		\centering
		\includegraphics[scale=0.58]{static/spectral_n1000_pin05_pout042.png}
		\caption{$n=1000$, $\Delta p=0.08$}
	\end{subfigure}
	\begin{subfigure}{.5\textwidth}
		\centering
		\includegraphics[scale=0.58]{static/spectral_n1000_pin02_pout07.png}
		\caption{$n=1000$, $\Delta p=-0.5$}
	\end{subfigure}
\end{figure}

La première observation que l'on peut faire est que les valeurs propres de nos matrices d'adjacence sont bien distribuées selon la loi du demi-cercle de Wigner.
De plus, en fonction de $\Delta p$ la mesure spectrale est perturbé (ou pas) par une ou deux valeurs propres qui sortent du support de la distribution initiale.\\

Dans les cas avec $\Delta p = 0.5$ on voit très clairement deux valeurs propres qui se détachent du support de la distribution de Wigner.
Les valeurs $z1$, $z2$ correspondent bien aux valeurs théoriques et dans ce cas on retrouve, avec un taux d'erreur de l'ordre de $10^{-2}$, les valeurs $p_{in}$, $p_{out}$ du modèle utilisé pour générer le graphe.\\
Dans les cas avec $\Delta p = -0.5$ on ne s'attend pas à ce que les valeurs théorique du modèles correspondent aux valeurs trouvées par la simulation dans la mesure où il n'y a en théorie aucune structure de communauté dans le graphe.
Cependant, on observe valeur propre en dehors du support de la loi de Wigner négative.
Par conséquent lorsque l'on obtient une valeur propre négative, le modèle spectral définit ci-dessus nous permet de conclure qu'il n'y a pas de structure de communauté dans le graphe.\\
Enfin, dans le cas avec $\Delta p = 0.08$, il y a en théorie une structure de communauté. 
Cependant on se retrouve dans le cas limite décrit dans \autoref{subsec:1.3}  

\section{Erreurs de l'article}
\subsection{Détermination de la mesure spectrale de X par la combinatoire}
Pour trouver la mesure spectrale de $X$, l'article utilise une méthode combinatoire.
Nous allons dérouler le raisonnement.\\
La transformé de Stieltjes de $X$ est :
\begin{equation}
	\rho(z) = \frac{1}{\pi} Im\langle Tr(zI - X)^{-1}\rangle
\end{equation}
où $\langle \dots \rangle$ indique la moyenne de l'ensemble.
On peut réécrire la trace de la moyenne comme ce qui suit: 
\begin{align}
	\langle Tr(zI - X)^{-1}\rangle &= \frac{1}{z}\sum_{k=0}^{\infty} \frac{Tr\langle X^k\rangle}{z^k} \\
	Tr\langle X^k\rangle &= \sum_{i_1\dots i_k}\langle X_{i_1i_2}X_{i_1i_2}\dots X_{i_ki_1}\rangle \label{eq: trace Xk}
\end{align}
On sait que $X$ est centrée et que les $X_{ij} \; \forall i\leq j$ sont des variables de Bernoulli indépendantes définies celon \eqref{eq: X}.
Par conséquent $\langle X_{i_1i_2}X_{i_1i_2}\dots X_{i_ki_1}\rangle \neq 0$ si $k = 2m$ avec $m\in \mathbb{N}$ et si chaque $X_{ij}$ apparaît exactement deux fois.
En effet
\begin{equation}
	\langle X_{ij}^2\rangle = \left\{
	\begin{array}{lr}
		\sigma_{in}^2  &\; si \; (i,j < \frac{n}{2}) \; ou \; (i,j \ge \frac{n}{2}) \\
		\sigma_{out}^2 &\; else \; where
	\end{array}
\right.\nonumber
\end{equation}

\eqref{eq: trace Xk}

\subsection{Normalisation de la mesure spectrale de X}
Dans l'article l'équation $(7)$ donne la mesure spectrale de la matrice $X$.
\begin{align*}
	\rho(z) &= \frac{n}{\pi} \frac{\sqrt{2(c_{in} + c_{out}) - z^2}}{c_{in} + c_{out}}
\end{align*}
Calculons l'intégrale de $\rho(z)$. 
On sait que son support est $[a, b] = [-\sqrt{2(c_{in} + c_{out})},\sqrt{2(c_{in} + c_{out})}]$.
\begin{align*}
	\int_{a}^{b} \rho(z) \, \mathrm{d}z &= \int_{a}^{b} \frac{n}{\pi} \frac{\sqrt{2(c_{in} + c_{out}) - z^2}}{c_{in} + c_{out}}  \, \mathrm{d}z \\
	&= \int_{a}^{b} \frac{1}{\pi} \frac{\sqrt{2 n (p_{in} + p_{out}) - z^2}}{p_{in} + p_{out}}  \, \mathrm{d}z \\
	&= \int_{a}^{b} \frac{\sqrt{n}}{\pi} \frac{\sqrt{2 (p_{in} + p_{out}) - \frac{z^2}{n}}}{p_{in} + p_{out}}  \, \mathrm{d}z  \\
	&= n \int_{a'}^{b'} \frac{1}{\pi} \frac{\sqrt{2 (p_{in} + p_{out}) - u^2}}{p_{in} + p_{out}}  \, \mathrm{d}u  
\end{align*}
avec $u^2 = \frac{z^2}{n}$, $a'= \sqrt{2(p_{in} + p_{out})}$ et $b'= \sqrt{2(p_{in} + p_{out})}$.
Or la fonction sous l’intégrale correspond à la loi du demi-cecle de Wigner de paramètre $\sigma = (p_{in} + p_{out})$, qui est normalisée sur le fermé $[a', b']$.
On obtient donc:
\begin{align*}
 	\int_{a}^{b} \rho(z) \, \mathrm{d}z &= n
\end{align*}


\section{Généralisation}
\subsection{Cas avec n communautés}
On peut a présent généraliser à un nombre de communautés $q \geq 2$.
Nous allons supposer que les communautés sont de même taille, à savoir $n_q = \frac{n}{q}$.
\paragraph{}\label{rq:contrainte model}
Une première contrainte apparaît, de par l'utilisation des théorèmes \ref{th:1} et \ref{th:2}, sur les valeurs des probabilités de la matrice d'adjacence $A$.
En effet, d'après le \autoref{th:1}, pour que la matrice $X$ est une mesure spectrale qui tende vers la loi de Wigner il faut que la norme 1 des vecteurs lignes de son profil de variance soient égales ($\parallel x \parallel_1 = \sum_{j=1}^{n}|x_j|$).
Par conséquent, si on veut augmenter le nombre de communautés $q$ dans le modèle, on est forcé de garder deux probabilités $p_{in}$ et $p_{out}$ qui jouent le même rôle que celles introduites précédemment (cf. \ref{rq:probability}).\\

On sait que la matrice d’adjacence du graphe sous le SBM à $q$ communautés est $A = X + \langle A \rangle$.  
Pour poursuivre l'analyse on va suivre la trame suivante:
\begin{itemize}
	\item[1-] Trouver l'équation de $\langle A \rangle$ ;
	\item[2-] Trouvez l'équation de X et déterminer son profil de variance ;
	\item[3-] Trouvez les $q$ valeurs propres associées aux perturbations de rang 1 ;
	\item[4-] Trouver $p_{lim}$.\\
\end{itemize}

% ------------------------------------------------- 1- trouver <A>  -------------------------------------------------
\subsubsection*{1- Équation de $\langle A \rangle$}
$\langle A \rangle$ étant symétrique, le théorème spectral nous dis qu'il existe une base orthonormée telle que $\langle A \rangle = \sum_{n}^{i=1}\lambda_i\mathbf{u}_i\mathbf{u}_i^{\star}$.
Après les calculs on trouve:
\begin{align} 
\langle A \rangle :&= n_q(p_{in} + (q-1)p_{out}) \mathbf{u}_1\mathbf{u}_1^{\star} + n_q(p_{in}-p_{out})\sum_{i=1}^{q-1}\mathbf{u}_i\mathbf{u}_i^{\star}\\
				   &= \frac{c_{in} + (q-1)c_{out}}{q} \mathbf{u}_1\mathbf{u}_1^{\star} + \frac{c_{in}-c_{out}}{q}\sum_{i=1}^{q-1}\mathbf{u}_i\mathbf{u}_i^{\star}
\end{align}\\
où les trois valeurs propres sont $0$, $n_q(p_{in} + (q-1)p_{out})$, $n_q(p_{in}-p_{out})$ de multiplicité $q(n_q - 1)$, $1$, $q-1$.\\

% ------------------------------------------------- 2- trouver X  -------------------------------------------------
\subsubsection*{2- Profil de variance de $\frac{X}{\sqrt{n}}$}
De la même manière que dans \autoref{ch:Analyse spectrale de la matrice d'adjacence} on trouve:
\begin{equation}
	X_{ij} \sim \left\{
	\begin{array}{lr}
		\sigma_{in} Z_{ij} & : (i,j \in P_{in}) \\
		\sigma_{out} Z_{ij} & : (i,j \in P_{out})
	\end{array}
\right.\nonumber
\end{equation}
Où $Z_{ij} = \frac{B_{ij}(p) - p}{\sqrt{p(1-p)}} \;\;avec \; p = p_{in} \; ou \; p_{out}$, $B_{ij}(p) \sim B(p)$, $B(p)$ loi de Bernoulli de paramètre p\\
La somme de n'importe quel vecteur ligne (ou colonne) du profil de variance de $\frac{X}{\sqrt{n}}$ est égale à : 
\begin{align}
\label{eq:sigma2} 
\sigma^2 = \frac{\sigma_{in}^2 + (q-1)\sigma_{out}^2}{q}
\end{align}
Le profil de variance de $\frac{X}{\sqrt{n}}$ est donc une matrice bi-stochastique.\\
% ------------------------------------------------- 3- trouver les vp -------------------------------------------------
\subsubsection*{3- Valeurs propres de $\frac{A}{\sqrt{n}}$}
Soit $\lambda$ une valeur propre de $\frac{A}{\sqrt{n}}$ et le vecteur propre associé.
\begin{align}
&\Leftrightarrow \frac{A}{\sqrt{n}}v =\lambda v \nonumber \\
&\Leftrightarrow (\Gamma - \lambda I)v =-\alpha \mathbf{11}^Tv - \beta \sum_{i=1}^{q-1}\mathbf{u}_i\mathbf{u'}_i^T \label{eq:generalize}
\end{align}
Pour trouver la valeur propre associée à $\mathbf{u}_1$ on multiplie à gauche par $\mathbf{u}_1^{\star}(\Gamma -\lambda I)^{-1}$ et on obtient:
\begin{align}
\eqref{eq:generalize} &\Leftrightarrow \mathbf{u}_1^{\star}v =-\alpha \mathbf{u}_1^{\star}(\Gamma -\lambda I)^{-1}\mathbf{u}_1\mathbf{u}_1^{\star}v - \beta \mathbf{u}_1^{\star}(\Gamma -\lambda I)^{-1}\sum_{i=1}^{q-1}\mathbf{u}_i\mathbf{u}_i^{\star}v \nonumber\\
&\xrightarrow[n \to +\infty]{} 1 = -\alpha g_{wig}^{\sigma^2}(\lambda) \nonumber\\
&\Leftrightarrow 1 = \alpha \frac{\lambda + \sqrt{\lambda^2 - 4\sigma^2}}{2\sigma^2} \nonumber\\
&\Leftrightarrow \lambda = \frac{c_{in} + (q-1)c_{out}}{q\sqrt{n}} + \frac{q\sqrt{n}\sigma^2}{c_{in} + (q-1)c_{out}} \label{eq:z_2 generalize}
\end{align}
Si on remplace $q$ par $2$ on retrouve l'équation \eqref{z_2}.\\
Pour trouver les valeurs propres associée aux $\mathbf{u}_i, \; \forall i \in \{2, \cdots, q-1\}$ on multiplie à gauche par $\mathbf{u}_2^{\star}(\Gamma -\lambda I)^{-1}$ et on obtient:
\begin{align}
\eqref{eq:generalize} &\Leftrightarrow \mathbf{u}_2^{\star}v =-\alpha \mathbf{u}_2^{\star}(\Gamma -\lambda I)^{-1}\mathbf{u}_1\mathbf{u}_1^{\star}v - \beta \mathbf{u}_2^{\star}(\Gamma -\lambda I)^{-1}\sum_{i=1}^{q-1}\mathbf{u}_i\mathbf{u}_i^{\star}v \nonumber\\
&\xrightarrow[n \to +\infty]{} 1 = -\beta g_{wig}^{\sigma^2}(\lambda) \nonumber\\
&\Leftrightarrow 1 = \beta \frac{\lambda + \sqrt{\lambda^2 - 4\sigma^2}}{2\sigma^2} \nonumber\\
&\Leftrightarrow \lambda = \frac{c_{in} - c_{out}}{q\sqrt{n}} + \frac{q\sqrt{n}\sigma^2}{c_{in} - c_{out}}\label{eq:z_1 generalize}
\end{align}
Si on remplace $q$ par $2$ on retrouve l'équation \eqref{z_1}.\\
Les valeurs propres de $A$ pour les valeurs propres de $\langle A \rangle$ égales à zéros appartiennent au support de la distribution de Wigner.
Elles n'apportent donc aucune information supplémentaire sur la structure de communauté du graphe étudié.\\   

Nous noterons $z_1 =$ \eqref{eq:z_1 generalize}  et $z_2 =$ \eqref{eq:z_2 generalize} pour la suite.
 

% ------------------------------------------------- 4- trouver p_lim -------------------------------------------------
\subsubsection*{4- Seuil de décidabilité $p_{lim}$}
Nous cherchons maintenant à déterminer $p_{lim}$.
La condition limite naturelle est celle où la valeur propre $z_2$ qui sort du support de la distribution de Wigner est égale au bord droit du support de la mesure spectrale de la matrice A.
On a alors 
\begin{align*}
	&\Leftrightarrow \lambda^+ = z_1\\
	&\Leftrightarrow 2 \sigma = \frac{c_{in} - c_{out}}{q\sqrt{n}} + \frac{q\sqrt{n}\sigma^2}{c_{in} - c_{out}}\\
	&\Leftrightarrow 0 = \beta \sigma^2 - 2 \sigma + \alpha \\
	&\Leftrightarrow p_{in} - p_{out} = \frac{q\sigma}{\sqrt{n}}  \\
\end{align*}
Donc
\begin{equation}
	p_{lim} = \frac{\sqrt{q(\sigma_{in}^2 + (q-1)\sigma_{out}^2)}}{\sqrt{n}} = \frac{q\sigma}{\sqrt{n}}  \\
\end{equation}


% ------------------------------------------------- 5- résumer -------------------------------------------------
\subsection{Bilan}
Ci-dessous le bilan de la généralisation:
\begin{align*}
	\sigma^2&: \frac{\sigma_{in}^2 + (q-1)\sigma_{out}^2}{q} \\
	z_1&: \frac{c_{in} - c_{out}}{q\sqrt{n}} + \frac{q\sqrt{n}\sigma^2}{c_{in} - c_{out}}\\
	z_2&: \frac{c_{in} + (q-1)c_{out}}{q\sqrt{n}} + \frac{q\sqrt{n}\sigma^2}{c_{in} + (q-1)c_{out}}\\
	p_{lim}&: \frac{\sqrt{q(\sigma_{in}^2 + (q-1)\sigma_{out}^2)}}{\sqrt{n}} \\
\end{align*}

% ------------------------------------------------- 6- Simulation -------------------------------------------------
\subsection{Simulations}
\begin{figure}[H]
\centering
\includegraphics[scale=0.6]{static/graph_q7_n100_pin08_pout0011.png}
\caption{Graphe généré à partir des paramètres: $q=7$ $n=100$, $p_{in}=0.8$, $p_{out}=0.01$}
\end{figure}
\begin{figure}[H]
\centering
\includegraphics[scale=0.6]{static/spectral_q4_n500_pin08_pout02}
\caption{$q=4$, $n=500$, $\Delta p=0.6$}
\label{n500delta-05}
\end{figure}
\subsection{Limites du modèle}
Le première limite de ce modèle est que l'on est cantonné à des communautés de même taille $n_q$.
En effet si on change ce paramètre pour chacune des communautés alors le \autoref{th:1} n'est plus applicable, la somme des éléments de chaque vecteurs ligne du profil de variance n'est plus constant.\\

La deuxième contrainte est le fait que l'on doit toujours garder deux paramètres $p_{in}$ et $p_{out}$ indépendamment du nombre de communautés $n_q$ et du nombre de nœuds dans le graphe $n$.
Idéalement nous souhaiterions avoir un paramètre $p_{ij}$ correspondant à la probabilité d'avoir une arrête entre le nœud $i$ et le nœud $j$ et ce $\forall i<j$.\\
Une manière d'encoder ces paramètres est d'utiliser la relation suivante $p_{ij} = q_iq_jC_{\alpha}$
\begin{align}
	p_{ij} &= q_iq_jC_{g_ig_j}
\end{align}
Où $q_i$ est la probabilité intrinsèque du nœud $i$ à avoir une arrête, $g_i$ est la communauté correspondant au nœud $i$ et $C_{g_ig_j}$ est le facteur de correction par communauté.\\
Cette formalisation plus générale est très répandu dans la théorie de la détection de communauté spectrale.

\subsection{Test complet de l’algorithme de clustering spectral de graphe}
Dans cette section nous allons retrouver les communautés d'un graphe généré à partir d'un SBM.
Pour ce faire, nous allons suivre la procédure expliquer en \ref{par:algo spectral clustering}.
Le score utilisé est le \textit{Rand index}, \cite[p.78 eq.(88)]{Community_detection_in_graphs}.
Pour catégoriser les nœuds projetés dans l'espace associé aux vecteurs propres nous avons utilisé l'algorithme \textit{K-means}.\\

Ci-dessous les batteries de tests, leur score ainsi que les nœuds projetés sur les plans des 4 valeurs propres les plus significatives
$\{z_0,z1\}$
$\{z_0,z2\}$
$\{z_0,z3\}$
$\{z_1,z2\}$
\begin{figure}[H]
	\begin{subfigure}{.5\textwidth}
		\centering
		\includegraphics[scale=0.58]{static/q3_nq9_pin09_pout02.png}
		\caption{$n=27$, $p_{in}=0.9$, $p_{out}=0.2$}
		\label{n27pin09pout02}
	\end{subfigure}
	\begin{subfigure}{.5\textwidth}
		\centering
		\includegraphics[scale=0.58]{static/q5_nq10_pin07_pout03.png}
		\caption{$n=50$, $p_{in}=0.7$, $p_{out}=0.3$}
		\label{n50pin07pout03}
	\end{subfigure}
	\begin{subfigure}{.5\textwidth}
		\centering
		\includegraphics[scale=0.58]{static/q3_nq9_pin06_pout03.png}
		\caption{$n=50$, $p_{in}=0.6$, $p_{out}=0.3$}
		\label{n50pin06pout03}
	\end{subfigure}
	\begin{subfigure}{.5\textwidth}
		\centering
		\includegraphics[scale=0.58]{static/q5_nq10_pin04_pout03.png}
		\caption{$n=50$, $p_{in}=0.4$, $p_{out}=0.3$}
		\label{n50pin04pout03}
	\end{subfigure}
\end{figure}

On remarque que même pour des graphes avec un $\Delta P$ très faible, la qualité du résultat est de l'ordre de $70\%$.
\nocite{*}

\part{Analyse d'algorithmes de graph spectral clusturing}
\section{Bethe Hessian}
\subsection{Principe}
Dans le contexte d'un SBM avec $n$ nœuds et $q$ communautés.
Chaque nœud appartient à une communauté $g_v \in \{1, \dots, q \}$
La probabilité d'une arrête entre une paire de nœud $(u, v)$ est $\mathbb{P}[A_{u,v} = 1] = p_{g_u,g_v}$.\\

L'algorithme est le même que pour la matrice d'adjacence $A$. 
Le modèle est suffisamment général pour prendre en compte un nombre arbitraire de communautés avec $q^2$ probabilités $p_{g_u,g_v}$ d’existence arrêtes différentes entre chaque communauté.\\

Il y a cependant l'apparition d'un paramètre de régularisation qu'il faut calculer au préalable.
Cependant il existe une formule fermée pour le calculer, à savoir $r= \sqrt{\frac{\langle d^2\rangle }{\langle d\rangle }- 1}$, où $\langle d\rangle$ et $\langle d^2\rangle$ sont le premier et le deuxième moment de la distribution des dégrée des nœuds dans le graphe.

L'avantage de cet opérateur linéaire par rapport à la matrice d'adjacence $A$ est, qu'avec le bon choix de $r$, l'ensemble des $q$ valeurs propres portant l'information de la structure de communauté du graphe sont négatives alors que toutes les valeurs propres du bulk de $H(r)$ sont positives.
Il suffit donc de calculer les $q$ valeurs propres négatives et s’arrêter dès que le signe change.
Grâce à ce critère discriminant, nous avons instantanément le nombre de communautés dans le graphe.
Dans le cas avec l'opérateur d'adjacence, il est impossible de faire une inférence du nombre de communauté dans la graphe.\\



Dans le cas la matrice Bethe Hessian est définie de la manière suivante:
\begin{equation}
	H(r) := (r^2 - 1)I - rA + D
\end{equation}
Où $D_{ii} = d_i \; \forall i \in V$ avec $d_i$ est le degré du nœud $i$, et $r$ est le paramètre de régularisation.\\
\subsection{Simulations}
Nous allons dans un premier temps simuler les densités spectrales des matrices Bethe hessiennes avec un SBM avec $q = 2$, $n_q=2000$, $p_{in} = 0.0035$, $p_{out} = 0.0005$.
Avec ces paramètres, le paramètre optimale est $r_{opt}\simeq2$.\\

On peut observer que lorsque $r$ est largement supérieur à $r_{opt}$, on perd la propriété où les valeurs propres contenant l'information sont négatives et celles du bulk sont positives.
Par conséquent, on ne peut plus trouver le nombre de communautés dans le graphe.
De même lorsque $r$ est largement inférieur à $r_{opt}$.

On peut voir que le comportement plus la densité spectrale en fonction de $r$ est le suivant:
\begin{itemize}
	\item plus r est grand plus la moyenne de la densité spectrale augmente
	\item plus r est proche de 0 plus les valeurs propres portant l'information se rapprochent de 0.
\end{itemize}

\begin{figure}[H]
	\begin{subfigure}{.5\textwidth}
		\centering
		\includegraphics[scale=0.4]{static/bh_5.png}
		\label{bh5}
	\end{subfigure}
	\begin{subfigure}{.5\textwidth}
		\centering
		\includegraphics[scale=0.4]{static/bh_3.png}
		\label{bh3}
	\end{subfigure}
	\begin{subfigure}{.5\textwidth}
		\centering
		\includegraphics[scale=0.4]{static/bh_2.png}
		\label{bh2}
	\end{subfigure}
	\begin{subfigure}{.5\textwidth}
		\centering
		\includegraphics[scale=0.4]{static/bh_1_5.png}
		\label{bh15}
	\end{subfigure}
	\begin{subfigure}{.5\textwidth}
		\centering
		\includegraphics[scale=0.4]{static/bh_1.png}
		\label{bh1}
	\end{subfigure}
	\begin{subfigure}{.5\textwidth}
		\centering
		\includegraphics[scale=0.4]{static/bh_0_5.png}
		\label{bh05}
	\end{subfigure}
\end{figure}

Comme décrite dans la première partie, il faut comparer la partition obtenue de la matrice Bethe Hessian avec celle obtenue via l'algorithme ``belief propagation''.
D'après le figure 2 du papier \cite{bethe_hessian} on a:
\begin{figure}[H]
	\centering
	\includegraphics[scale=0.5]{static/bh_results.png}
	\caption{graphe généré à partir d'un SBM: $n=10^5$}
	\label{bhres}
\end{figure}
Ce sur cette figure il y a en ordonnée la performance de chaque algorithme via une mesure du chevauchement.
La figure à gauche correspond au cas avec 2 communautés et une structure de communauté ``assortative''.
La figure au milieu correspond aux cas avec 2 communautés et une structure de communauté ``assortative''.
La figure à droite correspond au cas avec 3 communautés et une structure de communauté ``assortative''.


\bibliographystyle{plain}
\bibliography{b}
\end{document}
