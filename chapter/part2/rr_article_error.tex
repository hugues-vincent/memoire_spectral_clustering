\section{Erreurs de l'article}
Dans cette section, allons redémontrer les résultats obtenus dans le \autoref{tab:bilan} via la combinatoire.
Par la même occasion nous allons expliquer les erreurs commises dans l'article \cite[Graph spectra and the detectability of community structure in networks]{raj_rao}.
\subsection{Normalisation de la mesure spectrale de X}
Dans l'article l'équation $(7)$ donne la mesure spectrale de la matrice $X$.
\begin{align*}
	\rho(z) &= \frac{n}{\pi} \frac{\sqrt{2(c_{in} + c_{out}) - z^2}}{c_{in} + c_{out}}
\end{align*}
Calculons l'intégrale de $\rho(z)$. 
On sait que son support est $[a, b] = [-\sqrt{2(c_{in} + c_{out})},\sqrt{2(c_{in} + c_{out})}]$.
\begin{align*}
	\int_{a}^{b} \rho(z) \, \mathrm{d}z &= \int_{a}^{b} \frac{n}{\pi} \frac{\sqrt{2(c_{in} + c_{out}) - z^2}}{c_{in} + c_{out}}  \, \mathrm{d}z \\
	&= \int_{a}^{b} \frac{1}{\pi} \frac{\sqrt{2 n (p_{in} + p_{out}) - z^2}}{p_{in} + p_{out}}  \, \mathrm{d}z \\
	&= \int_{a}^{b} \frac{\sqrt{n}}{\pi} \frac{\sqrt{2 (p_{in} + p_{out}) - \frac{z^2}{n}}}{p_{in} + p_{out}}  \, \mathrm{d}z  \\
	&= n \int_{a'}^{b'} \frac{1}{\pi} \frac{\sqrt{2 (p_{in} + p_{out}) - u^2}}{p_{in} + p_{out}}  \, \mathrm{d}u  
\end{align*}
avec $u^2 = \frac{z^2}{n}$, $a'= \sqrt{2(p_{in} + p_{out})}$ et $b'= \sqrt{2(p_{in} + p_{out})}$.
Or la fonction sous l’intégrale correspond à la loi du demi-cercle de Wigner de paramètre $\sigma = (p_{in} + p_{out})$, qui est normalisée sur le fermé $[a', b']$.
On obtient donc:
\begin{align*}
 	\int_{a}^{b} \rho(z) \, \mathrm{d}z &= n
\end{align*}
Il y a donc un problème de normalisation.

\subsection{Choix de la variance des entrées de X}
Nous allons réécrire le raisonnement de l'article \cite[Graph spectra and the detectability of community structure in networks]{raj_rao} lorsque l'on souhaite déterminer la densité spectrale de $X$.\\

La transformé de Stieltjes de $X$ est :
\begin{equation}
	\rho(z) = \frac{1}{\pi} Im\langle Tr(zI - X)^{-1}\rangle
\end{equation}
où $\langle \dots \rangle$ indique la moyenne de l'ensemble.
On peut réécrire la trace de la moyenne comme ce qui suit: 
\begin{align}
	\langle Tr(zI - X)^{-1}\rangle &= \frac{1}{z}\sum_{k=0}^{\infty} \frac{Tr\langle X^k\rangle}{z^k} \\
	Tr\langle X^k\rangle &= \sum_{i_1\dots i_k}\langle X_{i_1i_2}X_{i_1i_2}\dots X_{i_ki_1}\rangle \label{eq: trace Xk}
\end{align}

D'après l'analyse préliminaire effectué \autoref{ch:Analyse spectrale de la matrice d'adjacence}, on sait que $X$ est centrée et que les $X_{ij} \; \forall i\leq j$ sont des variables de Bernoulli indépendantes définies celon \eqref{eq: X}.
Par conséquent $\langle X_{i_1i_2}X_{i_1i_2}\dots X_{i_ki_1}\rangle \neq 0$ si $k = 2m$ avec $m\in \mathbb{N}$ et si chaque $X_{ij}$ apparaît exactement deux fois.
De plus.
\begin{equation}
	\langle X_{ij}^2\rangle = \left\{
	\begin{array}{lr}
		\sigma_{in}^2  &\; si \; (i,j < \frac{n}{2}) \; ou \; (i,j \ge \frac{n}{2}) \\
		\sigma_{out}^2 &\; else \; where
	\end{array}
\right.\nonumber
\end{equation}

C'est à ce moment qu’apparaît l'erreur.  
En effet dans l'article ils choisissent comme variances des entrées de X: 
\begin{align*}
	\langle X_{ij}^2 \rangle = \frac{c_{in} + c_{out}}{2n} \;\; \forall \; i,j
\end{align*}
\subsection{Détermination de la mesure spectrale de X par la combinatoire}
Pour trouver la mesure spectrale de $X$, l'article utilise une méthode combinatoire.
Nous allons refaire le raisonnement en se basant sur la démonstration du théorème de Wigner par la combinatoire dans \cite[Introduction aux matrices aléatoires]{wigner}.\\