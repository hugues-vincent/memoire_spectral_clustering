Ce mémoire a été réalisé en collaboration avec Jamal Najim au sein de LIGM (Laboratoire d'Informatique Gaspard Monge) à l'Université de Marne La Vallée.
L'objectif de ce stage était d'étudier les méthodes de détection de communauté du point de vue des matrices aléatoires.\\

La détection de communautés est un important domaine de recherche. 
Depuis l'arrivé des nouvelles technologies et notamment depuis la propagation du Web, une quantité gigantesque d'information est apparue donnant ainsi naissance à une abondance de graphes et de réseaux à étudier.
Il y a un nombre substantiel d'applications à l'étude de ces graphes.
En effet la structure des graphes se retrouve partout: 
\begin{itemize}
	\item Ensemble des sites web ;
	\item Ensemble des articles scientifiques ;
	\item Organes du corps humain ;
	\item Théorèmes mathématiques ;
	\item Images ;
	\item Flux économiques.
\end{itemize}
Une propriété des réseaux que l'on souhaite naturellement dégager est leur structure de communauté.
Dit autrement, nous souhaitons trouver une partition des nœuds du graphe afin d'étudier les connections ou les similarités les éléments du réseaux en questions.\\

Dans ce contexte, la théorie des matrices aléatoires a une application direct, le ``spectral graph clustering''.
L'intérêt de ces méthodes de détection sont leur rapidité d’exécution.
C'est domaine de recherche récent qui date des années 2000 et qui à eu un regain de vigueur suite à la publication de l'article ``Graph spectra and the detectability of community structure in networks'' \cite{raj_rao} en 2012.

Dans ce mémoire, nous allons précisément étudier cet article.
Il y a une erreur théorique dans ce papier et nous allons essayer de redémontrer les résultats en dégageant l'erreur commise.
Ensuite nous étudierons des algorithmes spectral plus sophistiqués et performants.\\
