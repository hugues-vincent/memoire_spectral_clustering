\section*{Bilan}
Nous avons réussi à redémontrer les résultats de l'article \cite{raj_rao}.
L'intérêt de ces nouveaux résultats réside dans la nouvelle limite de cet algorithme de détection de communauté que nous avons trouvé.
En effet, cette limite est largement citée et réutilisée dans les articles de ``graphe spectral clustering''.

Une perspective serait de réétudier les quelques articles qui utilisent comme base l’ancienne limite et ainsi corriger les résultats dont ils dérivent.\\

Nous avons aussi étudié d'autres algorithmes spectraux plus fins doté d'un performance supérieur.
Les recherches récentes dans ce domaine se tournent principalement sur la ``Non-Backtracking Matrix''.
Ceci est dû à ces propriétés plus intéressantes que celle des autres opérateurs linéaires (e.g matrice d'adjacence).
Mais aussi et surtout grâce au fait qu'elle à des connections avec des théories en physique comme les modelés d'Ising.

\section*{Remerciement}
Je tiens à remercier Jamal Najim pour m'avoir aiguillé et soutenu tout au long de ce stage.
Ses conseils ont été essentiels à l'aboutissement de ce mémoire.

\newpage