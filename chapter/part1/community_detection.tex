\section{La détection de communauté}
\subsection{Motivations}
La théorie des réseaux, a pour but d'analyser les graphes correspondant à des réseaux tels que celui de l'Internet, de la politique ou de la classification en biologie.
Chacun de ces graphes ont des propriétés spécifiques telles que:
\begin{itemize}
 	\item[-] Centrality ;
 	\item[-] ``small world effect'' ;
 	\item[-] Clustering ;
 	\item[-] Efficiency ;
 	\item[-] Degree distribution ; 
 	\item[-] Community Structure.\\
 \end{itemize}
 C'est cette dernière propriété qui va nous intéresser, à savoir la structure de communauté d'un graphe.
 L'idée de communauté correspond à l'intuition selon laquelle il y a des nœuds qui ont un lien étroit et forment des sous ensembles de nœuds de ce graphe.
 Par exemple dans le graphe des représentants politique Français, les hommes politiques d'un même partie appartiendraient à une même communauté.
 Cependant, dans cet exemple les choses peuvent être plus subtiles que ça, c'est là qu'intervient la détection de communauté.

\subsection{Méthodes}
Il y a différentes classes de méthodes pour la détection de communauté, en voici une liste non exhaustive:
\begin{itemize}
	\item[-] Graph partitioning
	\item[-] Hierarchical clustering
	\item[-] Partitional clustering
	\item[-] Spectral clustering \\
\end{itemize}
\subsection{Perspectives}
Les limites des modèles spectraux qui sont en développement pour la détection de communauté sont les suivantes:
\begin{itemize}
	\item[1-] Seuil à partir duquel la méthode spectrale n'est plus en mesure de détecter le communauté ;  
	\item[2-] Temps de calcul sur de grands graphes;  
	\item[3-] Raffinement des modèles de communautés.  
\end{itemize}

\subsection{Comparaison des performances}
La notion de structure de communauté n'a pas de définition consensuelle.
Ceci est également vrai pour les métriques qui mesurent les performances des différentes méthode.
Ci-dessous une liste des métriques qui seront utilisées ultérieurement.
Pour une liste complète voir \cite[p.77-79]{Community_detection_in_graphs}.
\begin{itemize}
	\item[-] Fraction of correctly classified vertices ; 
	\item[-] Fowlkes and Mallows metric; 
	\item[-] Rand Index; 
	\item[-] Normalized Mutal Information; 
\end{itemize}