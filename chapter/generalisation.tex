\section{Généralisation}
\subsection{SBM avec 2 communautés de tailles différentes}

\subsection{SBM avec n communautés}

On peut a présent généraliser à un nombre de communautés $q \geq 2$.
Nous allons supposer que les communautés sont de même taille, à savoir $n_q = \frac{n}{q}$.
\paragraph{}\label{rq:contrainte model}
Une première contrainte apparaît, de par l'utilisation des théorèmes \ref{th:1} et \ref{th:2}, sur les valeurs des probabilités de la matrice d'adjacence $A$.
En effet, d'après le \autoref{th:1}, pour que la matrice $X$ est une mesure spectrale qui tende vers la loi de Wigner il faut que la norme 1 des vecteurs lignes de son profile de variance soient égales ($\parallel x \parallel_1 = \sum_{j=1}^{n}|x_j|$).
Par conséquent, si on veut augmenter le nombre de communautés $q$ dans le modèle, on est forcé de garder deux probabilités $p_{in}$ et $p_{out}$ qui jouent le même rôle que celles introduites précédemment (cf. \ref{rq:probability}).\\

On sait que la matrice d’adjacence du graphe sous le SBM à $q$ communautés est $A = X + \langle A \rangle$.  
Pour poursuivre l'analyse on va suivre la trame suivante:
\begin{itemize}
	\item[1-] Trouver l'équation de $\langle A \rangle$ ;
	\item[2-] Trouvez l'équation de X et déterminer son profile de variance ;
	\item[3-] Trouvez les $q$ valeurs propres associées aux perturbations de rang 1 ;
	\item[4-] Trouver $p_{lim}$.\\
\end{itemize}

% ------------------------------------------------- 1- trouver <A>  -------------------------------------------------
D'une manière générale on trouve que:
\begin{align} 
\langle A \rangle :&= n_q(p_{in} + (q-1)p_{out}) \mathbf{11}^T + n_q\frac{\sqrt{2(q-1)}}{\sqrt{q}}(p_{in}-p_{out})\sum_{i=1}^{q-1}\mathbf{u}_i\mathbf{u'}_i^T\\
				   &= \frac{c_{in} + (q-1)c_{out}}{q} \mathbf{11}^T + \frac{\sqrt{2(q-1)}}{q\sqrt{q}}(c_{in}-c_{out})\sum_{i=1}^{q-1}\mathbf{u}_i\mathbf{u'}_i^T
\end{align}
avec
\begin{align*}
\begin{bmatrix}
\mathbf{1} & \mathbf{u}_1 & \cdots & \mathbf{u}_{q - 1}
\end{bmatrix} 
&=
\begin{bmatrix}
\frac{\mathbf{v}_1}{\parallel \mathbf{v}_1 \parallel} & \frac{\mathbf{v}_2}{\parallel \mathbf{v}_2 \parallel} & \cdots & \frac{\mathbf{v}_q}{\parallel \mathbf{v}_q \parallel}
\end{bmatrix}\\
\begin{bmatrix}
\mathbf{1} & \mathbf{u'}_1 & \cdots & \mathbf{u'}_{q - 1}
\end{bmatrix} 
&=
\begin{bmatrix}
\frac{\mathbf{v'}_1}{\parallel \mathbf{v'}_1 \parallel} & \frac{\mathbf{v'}_2}{\parallel \mathbf{v'}_2 \parallel} & \cdots & \frac{\mathbf{v'}_q}{\parallel \mathbf{v'}_q \parallel}
\end{bmatrix}
\end{align*}
où 
\begin{align*}
\begin{bmatrix}
\mathbf{v}_1 & \mathbf{v}_2 &  \cdots & \mathbf{v}_{q}
\end{bmatrix} 
&=
\begin{bmatrix}
1 & -1 & -1 & \cdots & -1 \\
1 & 1 & 0 & \cdots & 0 \\
\vdots & 0 & \ddots &  & \vdots \\
\vdots & \vdots &  & \ddots & 0 \\
1 & 0 & \cdots& \cdots & 1 \\
\end{bmatrix} \otimes \mathbf{1}_{n_q} \in \mathit{M}_q \otimes \mathit{M}_{n_q\times1}=\mathit{M}_{n\times q}\\
\begin{bmatrix}
\mathbf{v'}_1 & \mathbf{v'}_2 &  \cdots & \mathbf{v'}_{q}
\end{bmatrix} 
&= \frac{1}{n}
\begin{bmatrix}
1 & -1 & -1 & \cdots & -1 \\
1 & q -1 & -1 & \cdots & -1 \\
\vdots & -1 & \ddots &  & \vdots \\
\vdots & \vdots &  & \ddots & -1 \\
1 & -1 & \cdots& \cdots & q-1 \\
\end{bmatrix}\otimes \mathbf{1}_{n_q} \in \mathit{M}_q \otimes \mathit{M}_{n_q\times1}=\mathit{M}_{n\times q}\\
\end{align*}
\begin{align}\label{eq:norm generalize}
\parallel \mathbf{v}_1 \parallel^2 &= \frac{1}{n} & \; & \langle \mathbf{1}  \:|\: \mathbf{u}_i  \rangle = 0 \;,\forall i\nonumber\\
\parallel \mathbf{v}_i \parallel^2 &= \frac{2n}{q} & \; & \langle \mathbf{1} \:|\: \mathbf{u'}_i  \rangle = 0 \;,\forall i \\
\parallel \mathbf{v'}_i \parallel^2 &= \frac{q - 1}{n} & \; & \langle \mathbf{u}_i\:|\: \mathbf{u'}_i  \rangle = \sqrt{\frac{q}{2(q-1)}} \;,\forall i\nonumber\\
 \langle \mathbf{u}_i \:|\: \mathbf{u}_j  \rangle &= \frac{1}{2n} \;,\forall i \neq j & \; & \langle \mathbf{u'}_i \:|\: \mathbf{u'}_j  \rangle = \frac{1}{q - 1} \;,\forall i \neq j \nonumber\\
\langle \mathbf{u}_i \:|\: \mathbf{u'}_j  \rangle &= 0 \;,\forall i \neq j \nonumber
\end{align}\\

% ------------------------------------------------- 2- trouver X  -------------------------------------------------
De la même manière que dans \autoref{ch:Analyse spectrale de la matrice d'adjacence} on trouve:
\begin{equation}
	X_{ij} \sim \left\{
	\begin{array}{lr}
		\sigma_{in} Z_{ij} & : (i,j \in P_{in}) \\
		\sigma_{out} Z_{ij} & : (i,j \in P_{out})
	\end{array}
\right.\nonumber
\end{equation}
Où $Z_{ij} = \frac{B(p) - p}{\sqrt{p(1-p)}} \;\;avec \; p = p_{in} \lor p_{out}$\\
Soit le profile de variance de $\frac{X}{\sqrt{n}}$ et $\sigma^2$ la somme de n'importe quel de ces vecteurs lignes on obtient: 
\begin{align}
\label{eq:sigma2} 
\sigma^2 = \frac{\sigma_{in}^2 + (q-1)\sigma_{out}^2}{q}
\end{align}\\

% ------------------------------------------------- 3- trouver les vp -------------------------------------------------
On va à présent essayer de trouver les $q$ valeurs propres de $A$ associées au perturbations de rang 1. 
Repartons de $A = X + \langle A \rangle$.
\begin{align}
&\Leftrightarrow \frac{A}{\sqrt{n}}v =\lambda v \nonumber \\
&\Leftrightarrow (\Gamma - \lambda I)v =-\alpha \mathbf{11}^Tv - \beta \sum_{i=1}^{q-1}\mathbf{u}_i\mathbf{u'}_i^T \label{eq:generalize}
\end{align}
Pour trouver la valeur propre associée à $\mathbf{11}^T$ on multiplie par à gauche par $\mathbf{1}^T(\Gamma -\lambda I)^{-1}$ et on obtient:
\begin{align}
\eqref{eq:generalize} &\Leftrightarrow \mathbf{1}^Tv =-\alpha \mathbf{1}^T(\Gamma -\lambda I)^{-1}\mathbf{11}^Tv - \beta \mathbf{1}^T(\Gamma -\lambda I)^{-1}\sum_{i=1}^{q-1}\mathbf{u}_i\mathbf{u'}_i^Tv \nonumber\\
&\xrightarrow[n \to +\infty]{} 1 = -\alpha g_{wig}^{\sigma^2}(\lambda) \nonumber\\
&\Leftrightarrow 1 = \alpha \frac{\lambda + \sqrt{\lambda^2 - 4\sigma^2}}{2\sigma^2} \nonumber\\
&\Leftrightarrow \lambda = \frac{c_{in} + (q-1)c_{out}}{q\sqrt{n}} + \frac{q\sqrt{n}\sigma^2}{c_{in} + (q-1)c_{out}} \label{eq:z2 generalize}
\end{align}
Si on remplace $q$ par $2$ on retrouve l'équation \eqref{z2}.\\
Pour trouver les valeurs propres associées aux $\mathbf{u}_i\mathbf{u'}_i^T$ on multiplie par à gauche par $\mathbf{u'}_i^T(\Gamma -\lambda I)^{-1}$ et on obtient:
\begin{align}
\eqref{eq:generalize} &\Leftrightarrow \mathbf{u'}_i^Tv =-\alpha \mathbf{u'}_i(\Gamma -\lambda I)^{-1}\mathbf{11}^Tv - \beta \mathbf{u'}_i(\Gamma -\lambda I)^{-1}\sum_{i=1}^{q-1}\mathbf{u}_i\mathbf{u'}_i^Tv \nonumber\\
&\xrightarrow[n \to +\infty]{} \mathbf{u'}_i^Tv = -\beta(\langle \mathbf{u'}_i\:|\: \mathbf{u}_i \rangle\mathbf{u'}_i^Tv + \sum_{j \neq i}^{} \langle \mathbf{u'}_i\:|\: \mathbf{u}_j \rangle\mathbf{u'}_j^Tv)g_{wig}^{\sigma^2}(\lambda) \nonumber\\
\eqref{eq:norm generalize} &\Rightarrow 1 = -\beta  \sqrt{\frac{q}{2(q-1)}} g_{wig}^{\sigma^2}(\lambda) \nonumber\\
&\Leftrightarrow 1 = \beta \sqrt{\frac{q}{2(q-1)}} \frac{\lambda + \sqrt{\lambda^2 - 4\sigma^2}}{2\sigma^2} \nonumber\\
&\Leftrightarrow \lambda = \frac{c_{in} - c_{out}}{q\sqrt{n}} + \frac{q\sqrt{n}\sigma^2}{c_{in} - c_{out}}\label{eq:z1 generalize}
\end{align}
Si on remplace $q$ par $2$ on retrouve l'équation \eqref{z1}.\\
Nous noterons $z1 =$ \eqref{eq:z1 generalize}  et $z2 =$ \eqref{eq:z2 generalize} pour la suite.
Ces deux équations suffisent à retrouver les probabilités $p_{in}$ et $p_{out}$ du modèle à partir des valeurs propres empiriques (i.e calculées numériquement sur la matrice adjacente du graphe étudier.)\\

% ------------------------------------------------- 4- trouver p_lim -------------------------------------------------
Nous cherchons maintenant à déterminer $p_{lim}$.
La condition limite naturelle est celle où la valeur propre maximale est égale au bord droit du support de la mesure spectrale de la matrice A.
On a alors 
\begin{align*}
	&\Leftrightarrow \lambda^+ = z1\\
	&\Leftrightarrow 2 \sigma = \frac{c_{in} - c_{out}}{q\sqrt{n}} + \frac{q\sqrt{n}\sigma^2}{c_{in} - c_{out}}\\
	&\Leftrightarrow 0 = \beta \sigma^2 - 2 \sigma + \alpha \\
	&\text{après résolution de l'équation on obtient une unique solution}\\
	&\Leftrightarrow p_{in} - p_{out} = \frac{q\sigma}{\sqrt{n}}  \\
\end{align*}
Donc
\begin{equation}
	p_{lim} = \frac{\sqrt{q(\sigma_{in}^2 + (q-1)\sigma_{out}^2)}}{\sqrt{n}} = \frac{q\sigma}{\sqrt{n}}  \\
\end{equation}


% ------------------------------------------------- 5- résumer -------------------------------------------------
Ci-dessous le bilan de la généralisation:
\begin{align*}
	\sigma^2&: \frac{\sigma_{in}^2 + (q-1)\sigma_{out}^2}{q} \\
	z1&: \frac{c_{in} - c_{out}}{q\sqrt{n}} + \frac{q\sqrt{n}\sigma^2}{c_{in} - c_{out}}\\
	z2&: \frac{c_{in} + (q-1)c_{out}}{q\sqrt{n}} + \frac{q\sqrt{n}\sigma^2}{c_{in} + (q-1)c_{out}}\\
	p_{lim}&: \frac{\sqrt{q(\sigma_{in}^2 + (q-1)\sigma_{out}^2)}}{\sqrt{n}} \\
\end{align*}
% ------------------------------------------------- 6- Simulation -------------------------------------------------

\begin{figure}[H]
\centering
\includegraphics[scale=0.6]{static/graph_q7_n100_pin08_pout0011.png.png}
\caption{Graphe généré à partir des paramètres: $q=7$ $n=100$, $p_{in}=0.8$, $p_{out}=0.01$}
\end{figure}