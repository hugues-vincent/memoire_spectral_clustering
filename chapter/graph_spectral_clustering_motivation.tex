\section{Motivations}
\subsection{Introduction}
La théorie des réseaux, à pour but d'analyser les graphes correspondants à des réseaux tels que celui de l'Internet, de la politique ou de la classification en biologie.
Chacun de ces graphes ont des propriétés spécifiques telles que:
\begin{itemize}
 	\item[-] Centrality ;
 	\item[-] ``small world effect'' ;
 	\item[-] Clustering ;
 	\item[-] Efficiency ;
 	\item[-] Degree distribution ; 
 	\item[-] Community Structure.\\
 \end{itemize}
 C'est cette dernière propriété qui va nous intéresser, à savoir la structure de communauté d'un graphe.
 L'idée de communauté correspond à l'intuition selon laquelle il y a des nœuds qui ont un lien étroit et forment des sous ensembles de nœuds de ce graphe.
 Par exemple dans le graphe des représentants politique Français, les hommes politiques d'un même partie appartiendrait à une même communauté.
 Cependant, dans cette exemple les choses peuvent être plus subtiles que ça, c'est là qu'intervient la détection de communauté.

\subsection{La détection de communauté}
Il y a différentes classes de méthodes pour la détection de communauté, en voici une liste non exhaustive:
\begin{itemize}
	\item[A -] Graph partitioning
	\item[B -] Hierarchical clustering
	\item[C -] Partitional clustering
	\item[D -] Spectral clustering \\
\end{itemize}

Les techniques qui nous intéressent sont celles du clustering spectrale.
L'idée centrale de ce type de techniques est qu'un graphe est représentable par une matrice à partir de laquelle on peut utiliser les théories relatives à l'algèbre linéaire.

Un graphe $G$ est la donnée d'un couple $(V, E)$ tel que $V$ est une ensemble de nœuds et $E$ et un ensemble d'arêtes (i.e un couple $(i,j)$ où $i,j \in V$).
A partir de cette définition on peut représenter la graphe par une matrice dont les éléments correspondent à certaines données de $G$.
Il existe toute un ensemble de matrices représentant le graphe:
\begin{itemize}
  	\item[-] A: Adjencency Matrix ;
  	\item[-] L: Laplacian Matrix ;
  	\item[-] M: Modularity Matrix ;
  	\item[-] H: Bethe Hessian Matrix ;
  	\item[-] D: "$\alpha$-normalized” Adjacency Matrix.\\
\end{itemize}
La matrice d'adjacence d'un graphe, notée $A$, est définie telle que $\forall i,j \in V, \; A_{ij} = \mathbbm{1}_{(i,j) \in E}$.
Une colonne $i$ représente le nœud $i$ dont chaque composante, $j$, est égale à $1$ si il existe une arête entre $i$ et $j$ et $0$ sinon.\\
La procédure pour associer une communauté aux nœuds de $G$ via une méthode spectrale est la suivante: 
\begin{itemize}
	\item[1-] Calcul des vecteurs propres, $v_i$, de la matrice de représentation du Graphe $G$ ;
	\item[2-] Sélection des $l$ vecteurs propres portant l'information de la structure de communauté ; 
	\item[3-] Construction de la matrice $W = [v_1, \cdots, v_l] \in \mathbb{R}^{n\times l}$ ; 
	\item[4-] Projection des vecteur lignes $r_j$ de $W$ sur l'espace de dimension $l$, où chaque $r_j$ correspond  au nœud $j$ ; 
	\item[5-] Catégorisation des vecteurs $r_j $ dans une communauté via des algorithmes de clustering: \textit{K-Means}, \textit{Expetation-Maximization}, \textit{Support vector machine }, etc.\\
\end{itemize}

\par{\underline{Étape 1:}}
La matrice $A$ est carré, par conséquent elle peut être interprétée comme la représentation d'un endomorphisme dans un espace X de dimension $n$ (nombre de nœuds dans $G$).
Soit la base canonique $(e_i)_{i=1:n}$, chaque $e_i$ correspond au nœud $i$.
Les vecteurs propres de $A$ sont donc des combinaisons linéaires des nœuds de $G$.
Les nœuds concernés ont donc une certaine dépendance.
\par{\underline{Étape 2:}}
Les entrées de la Matrice $A$ sont modélisées par des variables aléatoires vérifiant certaines hypothèses (e.g centrées, moments finis, indépendantes ...).
La théorie des matrices aléatoires nous dit qu’asymptotiquement la mesure spectrale de A converge vers une loi déterministe $\mu$ (loi qui dépend de la matrice étudiée).
Les valeurs propres de $A$ distribuées selon $\mu$ peuvent être interpréter comme des fluctuations de la simulation aléatoire.\\
Cependant, lorsque le graphe admet une structure de communauté, les entrées ne sont plus tout à fait indépendantes.
Dans ce cas de figure, la théorie prévoie que des valeurs propres sortent du support de la mesure spectrale $\mu$.
Ce sont ces valeurs propres qui contiennent l'information de la structure de communauté de $G$.\\
Une remarque importante à faire est la suivante, si la structure de communauté de $G$ n'est pas assez ``explicite'', les valeurs propres censées porter l'information de ces dernières ne sortiront pas du support de la mesure spectrale théorique, et donc, seront interprétées comme du simple bruit.
Dans cette situation, la méthode spectrale est incapable de déceler la structure de communauté de $G$.
\par{\underline{Étape 4:}}
Les vecteurs colonnes de W correspondent à des combinaisons linéaires de nœuds.
De par la construction A, les entrées $i$ de ces vecteurs colonnes sont la somme des arêtes entre ces nœuds en dépendance linaire et le nœud $i$.
Les vecteurs propres que l'on a filtré sont ceux qui portent l'information des communautés.
Donc chaque vecteur ligne représente un nœud du graphe et l'espace dans lequel il se meut constitue l'information de la structure de communauté. 