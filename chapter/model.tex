% \part{partie}
% \chapter{chapitre}
% \section{section}
% \subsection{sous-section}
% \subsubsection{sous-sous-section}
% \paragraph{paragraphe}
% \subparagraph{sous-paragraphe}
\section{Théorie}
\subsection{Contexte}
Le but de ce papier est d'expliquer en détail la méthode de détection de communauté via la théorie spectrale des graphes de l'article de RR. Nadakuditi et M. E. J. Newmann .

Nous allons nous placer dans le contexte d'un Stochastic Block Model (SBM) à deux communautés (i.e \textbf{q}=2).
Autrement dit, dans un graphe \textbf{G} non orienté à \textbf{n} noeuds dont la probabilité d'existence d'une arête entre deux noeuds est la suivante:
\begin{itemize}
    \item[Même communauté:] $\mathbf{p_{in}}$  
    \item[Différentes communautés:] $\mathbf{p_{out}}$\\
\end{itemize}

Nous noterons \textbf{A} la matrice d'adjacence du graphe \textbf{G}.
Elle est symétrique de par le fait que le graphe soit non orienté.
Nous supposerons d'ailleurs, que ses éléments sont rangés dans l'ordre de leur communauté.
Dans notre cas avec \textbf{q}=2, les $\mathbf{n}/2$ premières lignes correspondent aux noeuds de la communauté 1, et les $\mathbf{n}/2$ dernières la communauté 2.
Même chose pour les colonnes, par symétrie de \textbf{A}.
Chaque élément de la matrice \textbf{A} est simulé par une loi de Bernoulli avec: 
\begin{equation} 
 A_{ij} \sim \left\{
  \begin{array}{lr}
    B(p_{in}) & : (i,j < \frac{n}{2}) \lor (i,j \ge \frac{n}{2}) \\
    B(p_{out}) & : else \; where
  \end{array}
\right.\nonumber
\end{equation}

\subsection{Analyse spectrale de la matrice d'adjacence \textbf{A}}
L'idée générale de l'analyse spectrale qui va suivre est de nous ramener à un régime spectral de grande matrices aléatoires connu (ici )
